%compile with pdflatex on papeeria

\documentclass[a4paper,12pt]{article}


\usepackage{fontawesome}
\usepackage{fancyhdr}
\usepackage{fancyheadings}
\usepackage[ngerman,german]{babel}
\usepackage{german}
\usepackage[utf8]{inputenc}
%\usepackage[latin1]{inputenc}
\usepackage[active]{srcltx}
%\usepackage{svg}
%\usepackage{algorithm}
%\usepackage[noend]{algorithmic}
\usepackage{eurosym}
\usepackage{amsmath}
\usepackage{amssymb}
\usepackage{amsthm}
\usepackage{bbm}
\usepackage{enumerate}
\usepackage{graphicx}
\usepackage{ifthen}
\usepackage{listings}
\usepackage{enumitem}
%\usepackage{struktex}
\usepackage{hyperref}
\usepackage{tikz}
\usepackage{float}
\usepackage{subcaption}
\usepackage{array}
\captionsetup{compatibility=false}
\captionsetup[subfigure]{labelformat=empty}

\usepackage{pgfplots}
\pgfplotsset{compat=1.15}
\usepackage{mathrsfs}
\usetikzlibrary{arrows}

\definecolor{ccqqqq}{rgb}{0.8,0,0}
\definecolor{kolorwykresu}{rgb}{0.07,0.04,0.56}

\pagenumbering{gobble}

\usepackage{tabularray}
\usepackage{multirow}
\usepackage{booktabs,tabularx}

%\DeclareMathSymbol{\shortminus}{\mathbin}{AMSa}{"39}

\renewcommand\tabularxcolumn[1]{m{#1}}% for vertical centering text in X column

\newcolumntype{L}[1]{>{\raggedright\let\newline\\\arraybackslash\hspace{0pt}}m{#1}}
\newcolumntype{C}[1]{>{\centering\let\newline\\\arraybackslash\hspace{0pt}}m{#1}}
\newcolumntype{R}[1]{>{\raggedleft\let\newline\\\arraybackslash\hspace{0pt}}m{#1}}

\newcolumntype{Y}{>{\centering\arraybackslash}X}

%%%%%%%%%%%%%%%%%%%%%%%%%%%%%%%%%%%%%%%%%%%%%%%%%%%%%%
%%%%%%%%%%%%%% EDIT THIS PART %%%%%%%%%%%%%%%%%%%%%%%%
%%%%%%%%%%%%%%%%%%%%%%%%%%%%%%%%%%%%%%%%%%%%%%%%%%%%%%
\newcommand{\Fach}{1. Klausur aus der Mathematik}
\newcommand{\Name}{}
\newcommand{\datum}{}
\newcommand{\Matrikelnummer}{}
\newcommand{\Semester}{Q12}
\newcommand{\Uebungsblatt}{} %  <-- UPDATE ME
%%%%%%%%%%%%%%%%%%%%%%%%%%%%%%%%%%%%%%%%%%%%%%%%%%%%%%
%%%%%%%%%%%%%%%%%%%%%%%%%%%%%%%%%%%%%%%%%%%%%%%%%%%%%%

\setlength{\parindent}{0em}
\topmargin -1.0cm
\oddsidemargin 0cm
\evensidemargin 0cm
\setlength{\textheight}{9.2in}
\setlength{\textwidth}{6.0in}

%%%%%%%%%%%%%%%
%% Aufgaben-COMMAND
\newcommand{\Aufgabe}[1]{
  {
  \vspace*{0.5cm}
  \textsf{\textbf{Aufgabe #1}}
  \vspace*{0.2cm}
  
  }
}
%%%%%%%%%%%%%%
\hypersetup{
    pdftitle={\Fach{}: Übungsblatt \Uebungsblatt{}},
    pdfauthor={\Name},
    pdfborder={0 0 0}
}

\lstset{ %
language=java,
basicstyle=\footnotesize\tt,
showtabs=false,
tabsize=2,
captionpos=b,
breaklines=true,
extendedchars=true,
showstringspaces=false,
flexiblecolumns=true,
}

\newcommand*{\quadratbox}{\textbf{\fbox{\phantom{\huge{?}}}}}%

\title{Übungsblatt \Uebungsblatt{}}
\author{\Name{}}

\begin{document}

\fancyhead{}
\fancyhead[C]{\includegraphics[height=2.5cm]{lukasLogoV2.png}
\vspace{2cm}
}

\thispagestyle{fancy}

\lhead{
%\vspace{1cm}
  \sf \LARGE \Fach{} %\small \Name{} - \Matrikelnummer{}
}
\rhead{\sf \Semester{}   \datum{}}


\vspace*{0.2cm}

\vspace{4cm}
Alle Lösungen müssen mit Nebenrechnungen und Begründungen nachvollziehbar sein!

%\rhead{\sf \Semester{} }
\vspace*{0.2cm}

%\begin{center}
%%\LARGE \sf \textbf{Übungsblatt \Uebungsblatt{}}
%\end{center}
%\vspace*{0.2cm}

%%%%%%%%%%%%%%%%%%%%%%%%%%%%%%%%%%%%%%%%%%%%%%%%%%%%%%
%% Insert your solutions here %%%%%%%%%%%%%%%%%%%%%%%%
%%%%%%%%%%%%%%%%%%%%%%%%%%%%%%%%%%%%%%%%%%%%%%%%%%%%%%

\vspace{1cm}
  Name: \underline{\hspace{7cm}}
  \hfill
  Datum: \underline{\hspace{4cm}}

\vspace{0.8cm}

%\textbf{Hinweise:} Der Lösungsweg muss nachvollziehbar sein. Arbeitszeit \textbf{45 Minuten}.
%\vspace{0,5cm}Die Rechenwege müssen nachvollziehbar sein!
%
%\vspace{0,5cm} {TEIL A} - ohne Hilfsmittel - Bearbeitungszeit 30 Minuten
%\vspace {0.8cm}
% 
%GEOMETRIE


\begin{center}
  \begin{tblr}{
      width=1\linewidth,
      colspec = {Q[c,6em]Q[c,3em]Q[c,3em]Q[c,3em]Q[c,3em]Q[c,3em]Q[c,3em]Q[c,3em]Q[c,6em]Q[c,6em]},
      rowspec = {Q[m]Q[m]Q[m]Q[m]Q[m]Q[m]Q[m]Q[m]Q[m]Q[m]},
      colsep = 0mm,
      %row{1} = {2em,azure2,fg=white,font=\large\bfseries\sffamily},
      row{1} = {2em,font=\large\bfseries\sffamily},
      hlines, vlines,
    }
    \textbf{Aufgabe} & \textbf{1} & \textbf{2} & \textbf{3} & 
    \textbf{4} & \textbf{5} & \textbf{6}& \textbf{7}& \textbf{Gesamt} & \textbf{Note}\\
    {Mögliche \\ Punkte} & {4} & {6} & {3} & 7 & 23& 10 &8 & 61 \\
    {Erreichte \\ Punkte} &  &  &  &  & & & & & \\
  \end{tblr}
\end{center}

\vspace{0.3cm}
%\centerline{\huge\bfseries\sffamily Viel Erfolg !!!}
\newpage\null\thispagestyle{empty}\newpage
\newpage
\vspace{1,5cm} {TEIL A} - ohne Hilfsmittel. Bearbeitungszeit 40 min.
\vspace {0,2cm}

%\newpage
%\vspace*{-2cm}
\Aufgabe{1: (2BE+2BE)} %2BE

Untenstehende Abbildung zeigt den Graphen einer Funktion $f$.
\begin{center}
\includegraphics[width=10.789cm]{Q12_2Klausur20230301_1.jpg}
\end{center}
\begin{enumerate}[label={\alph*)}]
  \item Skizzieren Sie den Graphen der Ableitungsfunktion $f'(x)$ in der Abbildung.
  \item Skizzieren Sie den Graphen einer möglichen Stammfunktion $F(x)$ in der Abbildung.
\end{enumerate}

\Aufgabe{2: (1BE+2BE+1BE+2BE)} %1BE
Berechnen Sie jeweils den Wert des unbestimmten Integrals.

\begin{enumerate}[label={\alph*)}]
  \item $\int \frac{1}{2\sqrt{x}}\, dx$
  \item $\int x(1-x)^2\, dx$
  \item $\int \frac{e^x}{e}\, dx$
  \item $\int \frac{2x+4x^3}{x^2(1+x^2)}\, dx$
\end{enumerate}

\Aufgabe{3: (3BE)}
Bestimmen Sie jeweils den Wert von $a$ so, dass eine wahre Aussage ensteht.
\[ \int_0^a e^{2x}\, dx = \frac{e-1}{2} \]

\newpage
\Aufgabe{4: (2BE+5BE)} %+2BE
Beim Zoll stehen neun Personen an; vier von ihnen sind Schmuggler. Der Zollbeamte bittet drei dieser Personen zur Kontrolle.
  \begin{enumerate}[label={\alph*)}]
    \item Finden Sie eine passende Simulation durch ein Urnenmodell.
    \item Geben Sie jeweils einen passenden Term an, der berechnet mit welcher Wahrscheinlichkeit unter diesen drei Personen ...
      \begin{enumerate}[label={(\arabic*)}]
        \item kein Schmuggler
        \item genau ein Schmuggler
        \item höchstens ein Schmuggler
        \item mindestens ein Schmuggler
      \end{enumerate}
 ... ist.
  \end{enumerate}

\newpage
\vspace{1,5cm} {TEIL B} - mit Hilfsmitteln. Bearbeitungszeit 90 min.
\vspace {0,2cm}

\Aufgabe{5: (1BE+8BE+2BE+3BE+3BE+2BE+4BE)} %+6
Gegeben ist die Funktion $f: f(x) = \frac{x^3}{(x-1)^2};\, D_f=\mathbb{R}\backslash\{1\};$ ihr Graph ist $G_f$.
\begin{enumerate}[label={\alph*)}]
  \item Geben Sie die Schnittpunkte $G_f$ mit Koordinatenachsen an.
  \item Untersuchen Sie $G_f$ auf Extrempunkte (Lage und Art).
  \item Ermitteln Sie die Gleichung der Tangente $t$ an den $G_f$ im Punkt $(2|8)$.
  \item Gegeben ist die $f''(x)=\frac{6x}{(x-1)^4}$ (ohne Nachweis). Bestimmen Sie die Wendepunkte und Krümmungsverhalten von $G_f$.
  \item Zeichnen Sie $G_f$ und seine Asymptoten.
  \item Zeigen Sie, dass die Funktion $F: F(x) = 3 \ln |x-1| + \frac{1}{2}(x+2)^2 - \frac{1}{x-1};\, D_F=\mathbb{R}\backslash\{1\}$, eine Stammfunktion von $f$ ist.
  \item Der Graph $G_f$, die Gerade $g$ mit der Gleichung $y=x+2$ sowie die Geraden $h: x=2$ und $k: x=10$ beranden ein Flächenstück. Berechnen Sie seinen Flächeninhalt $A$.
\end{enumerate}

\Aufgabe{6: (3BE+2BE+3BE+2BE)}
Lungenatmung.\\
In einem medizinischen Fachbuch wird die momentane Änderungsrate des Luftvolumens in der Lunge eines Menschen durch die Funktion
\[f:f(t) = 2,0 \sin (\frac{2\pi}{5}\cdot t);\, D_f=\mathbb{R}_0^+,\] beschrieben.
Momentane Änderungsrate des Luftvolumens in der Lunge wird in Liter pro Sekunde ($\frac{l}{s}$) und die Zeit $t$ in Sekunden ($s$) angegeben.
\begin{enumerate}[label={\alph*)}]
  \item Ermitteln Sie $F(t)=\int_0^t f(x)\, dx$ und interpretieren Sie $F(t)$ in diesem Kontext.
  \item Skizzieren Sie die Graphen $G_F$ und $G_f$ der Funktionen $F$ und $f$ für $0 s \le t \le 10 s$.
  \item Erläutern Sie welches maximale Luftvolumen ein Mensch hiernach aufnehmen kann.
  \item Ein Patient atmet mit weniger Luftvolumen in der Lunge und kürzeren Abständen zwischen den Atemzügen. Finden Sie heraus, wie sich dadurch der Funktionsterm $f(t)$ ändert.
\end{enumerate}

\Aufgabe{7: (5BE+3BE)}
\begin{minipage}[t]{0.7\textwidth}
  Bei einem Glückspiel mit einem Glücksrad der abgebildeten Art soll der Erwartungswert der Auszahlung bei einmaligem Drehen 1,50\euro{} betragen.\\
  Die einzelnen Auszahlungsbeträge sind angegeben. Die Skizze ist nicht maßstabsgetreu.
  \begin{enumerate}[label={\alph*)}]
    \item Berechnen Sie, wie groß die Mittelpunktswinkel der beiden Sektoren gewählt werden müssen, die zu den Auszahlungen 0\euro{} und 4\euro{} gehören.
\\
\\
\\
ERSATZERGEBNIS $\alpha=40^{\circ}$.
    \item Bestimmen Sie die Standardabweichung der Zufallsgröße $X$, die die Auszahlung beschreibt, auf Cent genau.
  \end{enumerate}
\end{minipage}
\hspace*{0.75cm}
\begin{minipage}[t]{0.3\textwidth}
  \begin{figure}[H]
    \vspace{0cm}
    \centering
    \includegraphics[width=1\linewidth]{231202.png}
  \end{figure}
\end{minipage}

\vspace{2cm}
\centerline{Viel Erfolg \faThumbsOUp }

\newpage
{
\Huge{MUSTERLÖSUNG}
}

\small

\Aufgabe{1: (2BE+2BE)}

Untenstehende Abbildung zeigt den Graphen einer Funktion $f$.
\begin{center}
\includegraphics[width=10.789cm]{Q12_2Klausur20230301_1.jpg}
\end{center}

%\vspace{5cm}

\Aufgabe{2: (1BE+2BE+1BE+2BE)} 

\begin{enumerate}[label={\alph*)}]
  \item $\int \frac{1}{2\sqrt{x}}\, dx = \sqrt{x} +c$
  \item $\int x(1-x)^2\, dx = \int (x-2x^2+x^3)\, dx=\frac{x^2}{2}-\frac{2x^3}{3}+\frac{x^4}{4} +c $
  \item $\int \frac{e^x}{e}\, dx=\frac{1}{e} e^x +c$
  \item $\int \frac{2x+4x^3}{x^2(1+x^2)}\, dx=\ln |x^2+x^4|+c$
\end{enumerate}


\Aufgabe{3: (3BE)}
Bestimmen Sie jeweils den Wert von $a$ so, dass eine wahre Aussage ensteht.

$\int_0^a e^{2x}\, dx=[\frac{1}{2}e^{2x}]_0^a=\frac{1}{2}e^{2a}-\frac{1}{2}=\frac{e-1}{2}$;\\
$e^{2a}-1=e-1$;\\
$e^{2a}=e; a=\frac{1}{2}$

\newpage
\Aufgabe{4: (2BE+5BE)} %+2BE
\begin{enumerate}[label={\alph*)}]
  \item Urnenmodell:\\
    Aus einer Urne, in der sich (nur) vier schwarze und fünf weiße Kugeln befinden, werden drei Kugel ohne Zurücklegen gezogen.
  \begin{figure}[H]
    \vspace{0cm}
    \centering
    \includegraphics[width=1\linewidth]{231202_muster4.jpg}
  \end{figure}
  \item
      \begin{enumerate}[label={(\arabic*)}]
        \item $P(\text{,,kein Schmuggler''})=\frac{5}{9}\cdot\frac{4}{8}\cdot\frac{3}{7}\approx 11,9\%$
        \item $P(\text{,,genau ein Schmuggler''})=\frac{4}{9}\cdot\frac{5}{8}\cdot\frac{4}{7}+\frac{5}{9}\cdot\frac{4}{8}\cdot\frac{4}{7}\cdot\frac{5}{9}\cdot\frac{4}{8}\cdot\frac{4}{7}=\frac{10}{21}\approx 47,6\%$
        \item $P(\text{,,höchstens ein Schmuggler''})=\frac{5}{42}+\frac{10}{21}=\frac{25}{42}\approx 59,5\%$
        \item $P(\text{,,mindestens ein Schmuggler''})=1-P(\text{,,kein Schmuggler''})=\frac{37}{42}\approx 88,1\%$
      \end{enumerate}
\end{enumerate}

\newpage
\Aufgabe{5: (1BE+8BE+2BE+3BE+3BE+2BE+4BE)} %+6
Aufgabestellung geändert. Musterlösung passt nicht genau!
  \begin{figure}[H]
    \vspace{0cm}
    \centering
    \includegraphics[width=0.9\linewidth]{231202_muster5_1.jpg}
  \end{figure}
  \begin{figure}[H]
    \vspace{0cm}
    \centering
    \includegraphics[width=1\linewidth]{231202_muster5_2.jpg}
  \end{figure}

\Aufgabe{6: (3BE+2BE+3BE+2BE)}
  \begin{figure}[H]
    \vspace{0cm}
    \centering
    \includegraphics[width=0.8\linewidth]{231202_muster6_1.jpg}
  \end{figure}
  \begin{figure}[H]
    \vspace{0cm}
    \centering
    \includegraphics[width=0.8\linewidth]{231202_muster6_2.jpg}
  \end{figure}

\Aufgabe{7: (5BE+3BE)}
  \begin{figure}[H]
    \vspace{0cm}
    \centering
    \includegraphics[width=0.8\linewidth]{231202_muster7.jpg}
  \end{figure}

\end{document}
