\documentclass[a4paper,12pt]{article}
\usepackage{fancyhdr}
\usepackage{fancyheadings}
\usepackage[ngerman,german]{babel}
\usepackage{german}
\usepackage[utf8]{inputenc}
%\usepackage[latin1]{inputenc}
\usepackage[active]{srcltx}
%\usepackage{algorithm}
%\usepackage[noend]{algorithmic}
\usepackage{amsmath}
\usepackage{amssymb}
\usepackage{amsthm}
\usepackage{bbm}
\usepackage{enumerate}
\usepackage{graphicx}
\usepackage{ifthen}
\usepackage{listings}
%\usepackage{struktex}
\usepackage{hyperref}
\usepackage{enumitem}
%\usepackage{wrapfig}
\usepackage{float}
\usepackage{subcaption}
\captionsetup{compatibility=false}
\captionsetup[subfigure]{labelformat=empty}

\usepackage{pgfplots}
\pgfplotsset{compat=1.15}
\usepackage{mathrsfs}
\usetikzlibrary{arrows}

\definecolor{ccqqqq}{rgb}{0.8,0,0}

\pagenumbering{gobble}

%%%%%%%%%%%%%%%%%%%%%%%%%%%%%%%%%%%%%%%%%%%%%%%%%%%%%%
%%%%%%%%%%%%%% EDIT THIS PART %%%%%%%%%%%%%%%%%%%%%%%%
%%%%%%%%%%%%%%%%%%%%%%%%%%%%%%%%%%%%%%%%%%%%%%%%%%%%%%
\newcommand{\Fach}{2. Stegreifaufgabe aus der Mathematik}
\newcommand{\Name}{}
\newcommand{\datum}{}
\newcommand{\Matrikelnummer}{}
\newcommand{\Semester}{Q11}
\newcommand{\Uebungsblatt}{} %  <-- UPDATE ME
%%%%%%%%%%%%%%%%%%%%%%%%%%%%%%%%%%%%%%%%%%%%%%%%%%%%%%
%%%%%%%%%%%%%%%%%%%%%%%%%%%%%%%%%%%%%%%%%%%%%%%%%%%%%%

\setlength{\parindent}{0em}
\topmargin -1.0cm
\oddsidemargin 0cm
\evensidemargin 0cm
\setlength{\textheight}{9.2in}
\setlength{\textwidth}{6.0in}

%%%%%%%%%%%%%%%
%% Aufgaben-COMMAND
\newcommand{\Aufgabe}[1]{
  {
  \vspace*{0.5cm}
  \textsf{\textbf{Aufgabe #1}}
  \vspace*{0.2cm}
  
  }
}
%%%%%%%%%%%%%%
\hypersetup{
    pdftitle={\Fach{}: Übungsblatt \Uebungsblatt{}},
    pdfauthor={\Name},
    pdfborder={0 0 0}
}

\lstset{ %
language=java,
basicstyle=\footnotesize\tt,
showtabs=false,
tabsize=2,
captionpos=b,
breaklines=true,
extendedchars=true,
showstringspaces=false,
flexiblecolumns=true,
}

\title{Übungsblatt \Uebungsblatt{}}
\author{\Name{}}

\begin{document}
\thispagestyle{fancy}
\lhead{\sf \large \Fach{} \\ %\small \Name{} - \Matrikelnummer{}
}
\rhead{\sf \Semester{} \\  \datum{}}
\vspace*{0.2cm}
%\begin{center}
%%\LARGE \sf \textbf{Übungsblatt \Uebungsblatt{}}
%\end{center}
%\vspace*{0.2cm}

%%%%%%%%%%%%%%%%%%%%%%%%%%%%%%%%%%%%%%%%%%%%%%%%%%%%%%
%% Insert your solutions here %%%%%%%%%%%%%%%%%%%%%%%%
%%%%%%%%%%%%%%%%%%%%%%%%%%%%%%%%%%%%%%%%%%%%%%%%%%%%%%

Name: \underline{\hspace{7cm}}
  \hfill
  Datum: \underline{\hspace{4cm}}

%\vspace{1cm}
%\vspace{0.5cm}

\Aufgabe{1:}
   Bestimmen Sie in der Teilaufgabe a), b) und c) jeweils die Ableitungsfunktion:
 \begin{enumerate}[label={\alph*)}]
\item $f(x)=-2x^3-\frac{1}{3}x^2+2x-\frac{7}{2}$  
  \vspace{3cm}
\item $g(a)=3a^3x^2+x+\frac{1}{a^2}$
  \vspace{3cm}
\item $h(x)=\frac {\frac{1}{3}x^4 - 3x^2}{x^2+2}$
  \vspace{5cm}
\item Geben Sie die Gleichung eine möglichen Stammfunktion $F$ von $f$.
  \vspace{3cm}
 \item Vereinfachen Sie den Nenner und den Zähler der Funktion $h'(x)$ so weit, wie möglich.
\end{enumerate}

\begin{flushright}8BE \end{flushright}
\begin{flushright}Bitte wenden\end{flushright}

\newpage

\Aufgabe{2:}

Gegeben ist der gezeichnete Graph einer ganzrationalen Funktion $f$ (Bild 1)

\begin{figure}[!h]
  \centering
\begin{tikzpicture}[line cap=round,line join=round,>=triangle 45,x=1cm,y=1cm]
\begin{axis}[
x=1cm,y=1cm,
axis lines=middle,
ymajorgrids=true,
xmajorgrids=true,
xmin=-2.275044326241156,
xmax=2.8734485815603117,
ymin=-1.732567361219789,
ymax=2.547397177787344,
xtick={-2,-1,...,2},
ytick={-1,0,...,2},]
\clip(-2.275044326241156,-1.732567361219789) rectangle (2.8734485815603117,2.547397177787344);
\draw[line width=2pt,color=ccqqqq,smooth,samples=100,domain=-2.275044326241156:2.8734485815603117] plot(\x,{(\x)^(3)-2*(\x)^(2)-(\x)+1});
\begin{scriptsize}
\draw[color=ccqqqq] (-0.9498670212766045,-1.4863350917162403) node {$f$};
\end{scriptsize}
\end{axis}
\end{tikzpicture}
  \label{f1}
  \caption*{Bild 1}
\end{figure}

\begin{enumerate}[label={\alph*)}]
\item Skizzieren Sie zum gegebenen Graph der Funktion $f$  die zugehörige Ableitungsfunktion $f'$ (direkt im gegebenen Koordinatensystem) 
\item Betrachten Sie jetzt die weitere Bilder: Bild 2, Bild 3 und Bild 4. Welcher der dort dargestellten Graphen kann der Graph der Stammfunktion F von f sein?
Begründen Sie Ihre Entscheidung anhand von zwei verschiedenen Eigenschaften!
Geben Sie jeweils einen Grund an, warum die übrigen Graphen nicht die Stammfunktion von f darstellen können!

\end{enumerate}

\begin{figure}[H]
  \centering

    \subfloat[Bild 2]{
      \begin{tikzpicture}[scale=0.9, line cap=round,line join=round,>=triangle 45,x=1cm,y=1cm]
              \begin{axis}[
              x=1cm,y=1cm,
              axis lines=middle,
              ymajorgrids=true,
              xmajorgrids=true,
              xmin=-2.3019060283688253,
              xmax=2.9719414893617393,
              ymin=-1.8310602690212177,
              ymax=2.4489042699859294,
              xtick={-2,-1,...,2},
              ytick={-1,0,...,2},]
              \clip(-2.3019060283688253,-1.8310602690212177) rectangle (2.9719414893617393,2.4489042699859294);
              \draw[line width=2pt,color=ccqqqq,smooth,samples=100,domain=-2.3019060283688253:2.9719414893617393] plot(\x,{0-(1/4)*(\x)^(4)+2/3*(\x)^(3)-1/2*(\x)^(2)+(\x)});
              \begin{scriptsize}
              \draw[color=ccqqqq] (-0.69915780141845,-1.5848279995176686) node {$f$};
              \end{scriptsize}
              \end{axis}
        \end{tikzpicture}
    }
    \subfloat[Bild 3]{

          \begin{tikzpicture}[scale=0.9, line cap=round,line join=round,>=triangle 45,x=1cm,y=1cm]
          \begin{axis}[
          x=1cm,y=1cm,
          axis lines=middle,
          ymajorgrids=true,
          xmajorgrids=true,
          xmin=-2.301906028368825,
          xmax=2.9719414893617384,
          ymin=-1.831060269021217,
          ymax=2.4489042699859294,
          xtick={-2,-1,...,2},
          ytick={-1,0,...,2},]
          \clip(-2.301906028368825,-1.831060269021217) rectangle (2.9719414893617384,2.4489042699859294);
          \draw[line width=2pt,color=ccqqqq,smooth,samples=100,domain=-2.301906028368825:2.9719414893617384] plot(\x,{1/4*(\x)^(4)-2/3*(\x)^(3)-1/2*(\x)^(2)+(\x)});
          \begin{scriptsize}
          \draw[color=ccqqqq] (-1.5676861702127871,2.3638422132483394) node {$f$};
          \end{scriptsize}
          \end{axis}
          \end{tikzpicture}


    }      %<------------
    \subfloat[Bild 4]{

        \begin{tikzpicture}[scale=0.9, line cap=round,line join=round,>=triangle 45,x=1cm,y=1cm]
        \begin{axis}[
        x=1cm,y=1cm,
        axis lines=middle,
        ymajorgrids=true,
        xmajorgrids=true,
        xmin=-2.6600620567376185,
        xmax=2.488430851063855,
        ymin=-2.13549289313469,
        ymax=2.144471645872449,
        xtick={-2,-1,...,2},
        ytick={-1,0,...,2},]
        \clip(-2.6600620567376185,-2.13549289313469) rectangle (2.488430851063855,2.144471645872449);
        \draw[line width=2pt,color=ccqqqq,smooth,samples=100,domain=-2.6600620567376185:2.488430851063855] plot(\x,{0-(\x)^(3)-2*(\x)^(2)+(\x)+1});
        \begin{scriptsize}
        \draw[color=ccqqqq] (-2.4003989361702396,2.0594095891348596) node {$f$};
        \end{scriptsize}
        \end{axis}
        \end{tikzpicture}


    }


\end{figure}


\begin{flushright}8BE \end{flushright}

%%%%%%%%%%%%%%%%%%%%%%%%%%%%%%%%%%%%%%%%%%%%%%%%%%%%%%
%%%%%%%%%%%%%%%%%%%%%%%%%%%%%%%%%%%%%%%%%%%%%%%%%%%%%%
\end{document}
