\documentclass[a4paper,12pt]{article}
\usepackage{fancyhdr}
\usepackage{fancyheadings}
\usepackage[ngerman,german]{babel}
\usepackage{german}
\usepackage[utf8]{inputenc}
%\usepackage[latin1]{inputenc}
\usepackage[active]{srcltx}
%\usepackage{algorithm}
%\usepackage[noend]{algorithmic}
\usepackage{amsmath}
\usepackage{amssymb}
\usepackage{amsthm}
\usepackage{bbm}
\usepackage{enumerate}
\usepackage{graphicx}
\usepackage{ifthen}
\usepackage{listings}
%\usepackage{struktex}
\usepackage{hyperref}
\usepackage{enumitem}
%\usepackage{wrapfig}

\pagenumbering{gobble}

%%%%%%%%%%%%%%%%%%%%%%%%%%%%%%%%%%%%%%%%%%%%%%%%%%%%%%
%%%%%%%%%%%%%% EDIT THIS PART %%%%%%%%%%%%%%%%%%%%%%%%
%%%%%%%%%%%%%%%%%%%%%%%%%%%%%%%%%%%%%%%%%%%%%%%%%%%%%%
\newcommand{\Fach}{2. Stegreifaufgabe aus der Mathematik}
\newcommand{\Name}{}
\newcommand{\datum}{}
\newcommand{\Matrikelnummer}{}
\newcommand{\Semester}{Q11}
\newcommand{\Uebungsblatt}{} %  <-- UPDATE ME
%%%%%%%%%%%%%%%%%%%%%%%%%%%%%%%%%%%%%%%%%%%%%%%%%%%%%%
%%%%%%%%%%%%%%%%%%%%%%%%%%%%%%%%%%%%%%%%%%%%%%%%%%%%%%

\setlength{\parindent}{0em}
\topmargin -1.0cm
\oddsidemargin 0cm
\evensidemargin 0cm
\setlength{\textheight}{9.2in}
\setlength{\textwidth}{6.0in}

%%%%%%%%%%%%%%%
%% Aufgaben-COMMAND
\newcommand{\Aufgabe}[1]{
  {
  \vspace*{0.5cm}
  \textsf{\textbf{Aufgabe #1}}
  \vspace*{0.2cm}
  
  }
}
%%%%%%%%%%%%%%
\hypersetup{
    pdftitle={\Fach{}: Übungsblatt \Uebungsblatt{}},
    pdfauthor={\Name},
    pdfborder={0 0 0}
}

\lstset{ %
language=java,
basicstyle=\footnotesize\tt,
showtabs=false,
tabsize=2,
captionpos=b,
breaklines=true,
extendedchars=true,
showstringspaces=false,
flexiblecolumns=true,
}

\title{Übungsblatt \Uebungsblatt{}}
\author{\Name{}}

\begin{document}
\thispagestyle{fancy}
\lhead{\sf \large \Fach{} \\ %\small \Name{} - \Matrikelnummer{}
}
\rhead{\sf \Semester{} \\  \datum{}}
\vspace*{0.2cm}
%\begin{center}
%%\LARGE \sf \textbf{Übungsblatt \Uebungsblatt{}}
%\end{center}
%\vspace*{0.2cm}

%%%%%%%%%%%%%%%%%%%%%%%%%%%%%%%%%%%%%%%%%%%%%%%%%%%%%%
%% Insert your solutions here %%%%%%%%%%%%%%%%%%%%%%%%
%%%%%%%%%%%%%%%%%%%%%%%%%%%%%%%%%%%%%%%%%%%%%%%%%%%%%%

Name: \underline{\hspace{7cm}}
  \hfill
  Datum: \underline{\hspace{4cm}}

%\vspace{1cm}
%\vspace{0.5cm}

\Aufgabe{1:}
   Bestimmen Sie in der Teilaufgabe a), b) und c) jeweils die Ableitungsfunktion:
 \begin{enumerate}[label={\alph*)}]
\item $f(x)=-2x^3-\frac{1}{3}x^2+2x-\frac{7}{2}$  
\item $g(a)=3a^3x^2+x+\frac{1}{a^2}$
\item $h(x)=\frac {\frac{1}{3}x^4 - 3x^2}{x^2+2}$
\item Geben Sie die Gleichung eine möglichen Stammfunktion $F$ von $f$.
 \item Vereinfachen Sie den Nenner und den Zähler der Funktion $h'(x)$ so weit, wie möglich.
\end{enumerate}

\begin{flushright}8BE \end{flushright}

\Aufgabe{2:}

Gegeben ist der gezeichnete Graph einer ganzrationalen Funktion $f$ (Bild1)
\begin{enumerate}[label={\alph*)}]
\item Skizzieren Sie zum gegebenen Graph der Funktion $f$  die zugehörige Ableitungsfunktion $f'$ (direkt im gegebenen Koordinatensystem) 
\item Betrachten Sie jetzt die weitere Bilder: Bild 2, Bild 3 und Bild 4. Welcher der dort dargestellten Graphen kann der Graph der Stammfunktion F von f sein?
Begründen Sie Ihre Entscheidung anhand von zwei verschiedenen Eigenschaften!
Geben Sie jeweils einen Grund an, warum die übrigen Graphen nicht die Stammfunktion von f darstellen können!

\end{enumerate}





\begin{flushright}8BE \end{flushright}

%%%%%%%%%%%%%%%%%%%%%%%%%%%%%%%%%%%%%%%%%%%%%%%%%%%%%%
%%%%%%%%%%%%%%%%%%%%%%%%%%%%%%%%%%%%%%%%%%%%%%%%%%%%%%
\end{document}
