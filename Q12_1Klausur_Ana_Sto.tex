%compile with pdflatex on papeeria

\documentclass[a4paper,12pt]{article}
\usepackage{fancyhdr}
%\usepackage{fancyheadings}
\usepackage[ngerman,german]{babel}
\usepackage{german}
\usepackage[utf8]{inputenc}
%\usepackage[latin1]{inputenc}
\usepackage[active]{srcltx}
%\usepackage{algorithm}
%\usepackage[noend]{algorithmic}
\usepackage{amsmath}
\usepackage{amssymb}
\usepackage{amsthm}
\usepackage{bbm}
\usepackage{enumerate}
\usepackage{graphicx}
\usepackage{ifthen}
\usepackage{listings}
\usepackage{enumitem}
%\usepackage{struktex}
\usepackage{hyperref}
\usepackage{tikz}
\usepackage{float}
\usepackage{subcaption}
\captionsetup{compatibility=false}
\captionsetup[subfigure]{labelformat=empty}

\usepackage{pgfplots}
\usepgfplotslibrary{fillbetween}
%\usetikzlibrary{patterns}
\pgfplotsset{compat=1.15}
\usepackage{mathrsfs}
\usetikzlibrary{arrows}

\pgfplotsset{grid style={dashed,gray}}

\definecolor{kolorwykresu}{rgb}{0.07,0.04,0.56}

\pagenumbering{gobble}

%%%%%%%%%%%%%%%%%%%%%%%%%%%%%%%%%%%%%%%%%%%%%%%%%%%%%%
%%%%%%%%%%%%%% EDIT THIS PART %%%%%%%%%%%%%%%%%%%%%%%%
%%%%%%%%%%%%%%%%%%%%%%%%%%%%%%%%%%%%%%%%%%%%%%%%%%%%%%
\newcommand{\Fach}{1. Klausur aus der Mathematik (A)}
\newcommand{\Name}{}
\newcommand{\datum}{}
\newcommand{\Matrikelnummer}{}
\newcommand{\Semester}{Q12/1}
\newcommand{\Uebungsblatt}{} %  <-- UPDATE ME
%%%%%%%%%%%%%%%%%%%%%%%%%%%%%%%%%%%%%%%%%%%%%%%%%%%%%%
%%%%%%%%%%%%%%%%%%%%%%%%%%%%%%%%%%%%%%%%%%%%%%%%%%%%%%

\setlength{\parindent}{0em}
\topmargin -1.0cm
\oddsidemargin 0cm
\evensidemargin 0cm
\setlength{\textheight}{9.2in}
\setlength{\textwidth}{6.0in}

%%%%%%%%%%%%%%%
%% Aufgaben-COMMAND
\newcommand{\Aufgabe}[1]{
  {
  \vspace*{0.5cm}
  \textsf{\textbf{Aufgabe #1}}
  \vspace*{0.2cm}
  
  }
}
%%%%%%%%%%%%%%
\hypersetup{
    pdftitle={\Fach{}: Übungsblatt \Uebungsblatt{}},
    pdfauthor={\Name},
    pdfborder={0 0 0}
}

\lstset{ %
language=java,
basicstyle=\footnotesize\tt,
showtabs=false,
tabsize=2,
captionpos=b,
breaklines=true,
extendedchars=true,
showstringspaces=false,
flexiblecolumns=true,
}

\title{Übungsblatt \Uebungsblatt{}}
\author{\Name{}}

\begin{document}
\thispagestyle{fancy}
%\lhead{\sf \large \Fach{} \\ %\small \Name{} - \Matrikelnummer{}
\lhead{\sf \large \Fach{} %\small \Name{} - \Matrikelnummer{}
}
\rhead{\sf \Semester{}   \datum{}}
%\rhead{\sf \Semester{} }
\vspace*{0.2cm}
%\begin{center}
%%\LARGE \sf \textbf{Übungsblatt \Uebungsblatt{}}
%\end{center}
%\vspace*{0.2cm}

%%%%%%%%%%%%%%%%%%%%%%%%%%%%%%%%%%%%%%%%%%%%%%%%%%%%%%
%% Insert your solutions here %%%%%%%%%%%%%%%%%%%%%%%%
%%%%%%%%%%%%%%%%%%%%%%%%%%%%%%%%%%%%%%%%%%%%%%%%%%%%%%

  Name: \underline{\hspace{7cm}}
%\draw[line width=1pt,color=ccqqqq,smooth,samples=100,domain=-6:7] plot(\x,{(1/12)*\x*\x+(1/3)*\x});
  \hfill
  Datum: \underline{\hspace{4cm}}

%\vspace{0,5cm}Die Rechenwege müssen nachvollziehbar sein!

%\vspace{1,5cm} {TEIL A} - ohne Hilfsmittel. Bearbeitungszeit 35 min.
\vspace {2cm}
 

\Aufgabe{1:} 
Gegeben ist der Graph der Funktion $f$. Bestimmen Sie näherungsweise den Wert des Integrals ${\int_{0}^{10} f(x)\, dx}$.\\
Ermitteln Sie die prozentuale Abweichung von dem realen Wert von 51.
\begin{figure}[H]
  \centering
  \includegraphics[width=0.5\columnwidth]{211117_8.png}
  %\caption{A boat.}
  %\label{fig:boat1}
\end{figure}


\vspace{1cm}

\Aufgabe{2:} 
\begin{enumerate}[label={\alph*)}] 
  \item Berechnen Sie den Wert des Integrals:
    \[ \int_{3}^{5} \frac{x+1}{2\sqrt{0,5x^2+x}}\,dx \]
  \item Bestimmen Sie den Inhalt der Fläche, die die Graphen der Funktionen ${f(x)= \frac{1}{4}x^2}$ und ${g(x) = x+3}$ mit der $y$-Achse einschließen. (Skizze!)
\end{enumerate}

\vspace{1cm}

\Aufgabe{3:} 
Bestimmen Sie die Nullstellen, Extrema, Wendepunkte und das Krümmungsverhalten des Graphen der Funktion ${f(x)=x\cdot e^{-x}}$.

\newpage

\Aufgabe{4:} 
\begin{enumerate}[label={\alph*)}] 
  \item Bestimmen Sie eine ganzrationale Funktion höchstens fünften Grades, deren Graph bzgl. der $y$-Achse symmetrisch ist und in $P (-1|1)$ eine Wendetangente mit der Steigung 3 hat.
  \item Geben Sie einen Funktionsterm einer Funktion $f$ an, dessen Graph streng monoton steigt und rechtsgekrümmt ist.
\end{enumerate}


\Aufgabe{5:} 
In einer Urne sind Kugeln mit den Buchstaben: $AGKMOLNES$
\begin{enumerate}[label={\alph*)}] 
  \item Es wird mit Zurücklegen gezogen. Mit welcher Wahrscheinlichkeit wird das Wort $KAMEL$ in dieser Reihenfolge gezogen?
  \item Es wird ohne Zurücklegen gezogen. Mit welcher Wahrscheinlichkeit kommt jetzt das Wort $KAMEL$ in dieser Reihenfolge?
  \item Welche Wahrscheinlichkeit erhält man jeweils wenn das Wort bei a) und b) $KAMELE$ ist?
\end{enumerate}


\Aufgabe{6:} 
Bei einem Basketballtraining sollen die Teilnehmer Korbleger mit links üben. Sepps Trefferwahrscheinlichkeit liegt bei 0,4.
\begin{enumerate}[label={\alph*)}] 
  \item Bestimmen Sie die Wahrscheinlichkeiten dafür, dass Sepp bei 5 Würfen\\
    i) mindestens 2 Körbe trifft,  ii) genau beim ersten und beim letzten Wurf trifft.
  \item Wie oft muss Sepp mindestens werfen, damit er mit einer Wahrscheinlichkeit von 98\% wenigstens einmal trifft?
\end{enumerate}



\Aufgabe{7:} 
Anna behauptet, sie sei Weinkennerin und könne das Ursprungsland eines Weines am Geschmack erkennen.
\begin{enumerate}[label={\alph*)}] 
  \item Wenn wir davon ausgehen, dass Anna eine Trefferquote von 80\% hat, wie hoch ist die Wahrscheinlichkeit, dass sie das Ursprungsland\\
    i) bei genau 3 von 5 Weinen bzw. 
    ii) bei 3 von 5 Weinen richtig bestimmt? (3)
  \item Wie groß muss ihre Trefferquote mindestens sein, damit die Wahrscheinlichkeit dafür, dass sie von 5 Weinen mindestens ein Ursprungsland richtig bestimmt, mindestens 92,0\% beträgt? (4)
\end{enumerate}

%\vspace{2cm}

\centerline{Viel Erfolg}


\end{document}
