%compile with pdflatex on papeeria

\documentclass[a4paper,12pt]{article}
\usepackage{fancyhdr}
%\usepackage{fancyheadings}
\usepackage[ngerman,german]{babel}
\usepackage{german}
\usepackage[utf8]{inputenc}
%\usepackage[latin1]{inputenc}
\usepackage[active]{srcltx}
%\usepackage{algorithm}
%\usepackage[noend]{algorithmic}
\usepackage{amsmath}
\usepackage{amssymb}
\usepackage{amsthm}
\usepackage{bbm}
\usepackage{enumerate}
\usepackage{graphicx}
\usepackage{ifthen}
\usepackage{listings}
\usepackage{enumitem}
%\usepackage{struktex}
\usepackage{hyperref}
\usepackage{tikz}
\usepackage{float}
\usepackage{subcaption}
\captionsetup{compatibility=false}
\captionsetup[subfigure]{labelformat=empty}

\usepackage{pgfplots}
\usepgfplotslibrary{fillbetween}
%\usetikzlibrary{patterns}
\pgfplotsset{compat=1.15}
\usepackage{mathrsfs}
\usetikzlibrary{arrows}

\pgfplotsset{grid style={dashed,gray}}

\definecolor{kolorwykresu}{rgb}{0.07,0.04,0.56}

\pagenumbering{gobble}

%%%%%%%%%%%%%%%%%%%%%%%%%%%%%%%%%%%%%%%%%%%%%%%%%%%%%%
%%%%%%%%%%%%%% EDIT THIS PART %%%%%%%%%%%%%%%%%%%%%%%%
%%%%%%%%%%%%%%%%%%%%%%%%%%%%%%%%%%%%%%%%%%%%%%%%%%%%%%
\newcommand{\Fach}{2. Klausur aus der Mathematik (A)}
\newcommand{\Name}{}
\newcommand{\datum}{}
\newcommand{\Matrikelnummer}{}
\newcommand{\Semester}{Q12/2}
\newcommand{\Uebungsblatt}{} %  <-- UPDATE ME
%%%%%%%%%%%%%%%%%%%%%%%%%%%%%%%%%%%%%%%%%%%%%%%%%%%%%%
%%%%%%%%%%%%%%%%%%%%%%%%%%%%%%%%%%%%%%%%%%%%%%%%%%%%%%

\setlength{\parindent}{0em}
\topmargin -1.0cm
\oddsidemargin 0cm
\evensidemargin 0cm
\setlength{\textheight}{9.2in}
\setlength{\textwidth}{6.0in}

%%%%%%%%%%%%%%%
%% Aufgaben-COMMAND
%\newcommand{\Aufgabe}[1]{
%  {
%  \vspace*{0.5cm}
%  \textsf{\textbf{Aufgabe #1}}
%  \vspace*{0.2cm}
%  
%  }
%}
%
\newcounter{aufgabencounter}
\newcommand{\aufgabeNr}{\stepcounter{aufgabencounter}{\theaufgabencounter}}


%%%%%%%%%%%%%%%
%% Aufgaben-COMMAND
\newcommand{\Aufgabe}[1]{
  {
  \vspace*{0.5cm}
  \textsf{\textbf{Aufgabe \aufgabeNr} #1}
  \vspace*{0.2cm}
  
  }
}


%%%%%%%%%%%%%%
\hypersetup{
    pdftitle={\Fach{}: Übungsblatt \Uebungsblatt{}},
    pdfauthor={\Name},
    pdfborder={0 0 0}
}

\lstset{ %
language=java,
basicstyle=\footnotesize\tt,
showtabs=false,
tabsize=2,
captionpos=b,
breaklines=true,
extendedchars=true,
showstringspaces=false,
flexiblecolumns=true,
}

\title{Übungsblatt \Uebungsblatt{}}
\author{\Name{}}

\begin{document}
\thispagestyle{fancy}
%\lhead{\sf \large \Fach{} \\ %\small \Name{} - \Matrikelnummer{}
\lhead{\sf \large \Fach{} %\small \Name{} - \Matrikelnummer{}
}
\rhead{\sf \Semester{}   \datum{}}
%\rhead{\sf \Semester{} }
\vspace*{0.2cm}
%\begin{center}
%%\LARGE \sf \textbf{Übungsblatt \Uebungsblatt{}}
%\end{center}
%\vspace*{0.2cm}

%%%%%%%%%%%%%%%%%%%%%%%%%%%%%%%%%%%%%%%%%%%%%%%%%%%%%%
%% Insert your solutions here %%%%%%%%%%%%%%%%%%%%%%%%
%%%%%%%%%%%%%%%%%%%%%%%%%%%%%%%%%%%%%%%%%%%%%%%%%%%%%%

  Name: \underline{\hspace{7cm}}
%\draw[line width=1pt,color=ccqqqq,smooth,samples=100,domain=-6:7] plot(\x,{(1/12)*\x*\x+(1/3)*\x});
  \hfill
  Datum: \underline{\hspace{4cm}}

%\vspace{0,5cm}Die Rechenwege müssen nachvollziehbar sein!


\vspace{1,5cm} 
\begin{center}
  {\bf TEIL A} - Analysis (ohne Hilfsmittel)
\end{center}
\vspace {0,2cm}
 

\Aufgabe{}
Gegeben ist die Funktion $f:f(x) = 2 \cdot \sqrt{x+4} -3$.

\begin{enumerate}[label={\alph*)}]
 \item Bestimmen Sie die Definitionsmenge $D_f$ der Funktion $f$.
  \begin{flushright}2 BE \end{flushright}
 \item Berechnen Sie die Schnittpunkte des Graphen $G_f$ mit den Koordinatenachsen.
  \begin{flushright}5 BE \end{flushright}
\end{enumerate}

\vspace{1cm}

\Aufgabe{}
Gegeben ist die Funktion $f: x \rightarrow \frac{\ln x}{x}, D_f = \mathbb{R}^{+}$.
\begin{enumerate}[label={\alph*)}]
 \item Berechnen Sie die Nullstelle sowie mögliche Extremwerte der Funktion.
  \begin{flushright}7 BE \end{flushright}
    [Zur Kontrolle: $f'(x) = \frac{2\ln x - 3}{x^3}$]
 \item Ermitteln Sie die Koordinaten des Wendepunkts $W$ von $G_f$.
  \begin{flushright}5 BE \end{flushright}
\end{enumerate}


\newpage
\begin{center}
  {\bf TEIL B} - Stochastik
\end{center}
\vspace {0,2cm}

{\it Runden Sie die Wahrscheinlichkeit jeweils auf zwei Prozentdezimalen.}\\

\Aufgabe{}
Ein Glücksrad hat vier gleich große Sektoren, in denen jeweils die Zahlen 0; 2; 4 und 8 stehen. Die Zahlen geben auch gleichzeitig an, wie viele Euro pro Spiel ausgezahlt werden.
\begin{enumerate}[label={\alph*)}]
  \item Berechnen Sie den Erwartungswert!
  \begin{flushright}3 BE \end{flushright}
  \item Der Einsatz bei einem fairen Spiel beträgt 2,50€. Begründen Sie, wie man die Sektoren verändern müsste, um das Spiel „fair“ zu machen!
  \begin{flushright}2 BE \end{flushright}
\end{enumerate}

\Aufgabe{}
Bei einer Umfrage unter den Teilnehmern der Mathekurse Q12 befürchten 23\% der SchülerInnen, in der schriftlichen Abiturprüfung eine Minderleistung zu schreiben. Der Englischlehrer unterhält sich mit fünf Personen des Abiturjahrgangs.
\begin{enumerate}[label={\alph*)}]
  \item Berechnen Sie die Wahrscheinlichkeit, mit der vier dieser fünf Personen befürchten,
im Matheabitur eine Minderleistung zu schreiben!
  \begin{flushright}2 BE \end{flushright}
\item Berechnen Sie die Wahrscheinlichkeit, mit der höchstens ein Teilnehmer befürchtet, eine Minderleistung zu erzielen!
  \begin{flushright}3 BE \end{flushright}
\end{enumerate}

\vspace{2cm}
\centerline{Viel Erfolg}


%\enlargethispage{2\baselineskip}

%\addtolength{\voffset}{-2cm}




%\begin{tikzpicture}
%\draw [very thin, black, step=0.5cm] (0,0) grid +(15,18);
%\end{tikzpicture}




%%%%%%%%%%%%%%%%%%%%%%%%%%%%%%%%%%%%%%%%%%%%%%%%%%%%%%
%%%%%%%%%%%%%%%%%%%%%%%%%%%%%%%%%%%%%%%%%%%%%%%%%%%%%%
\end{document}
