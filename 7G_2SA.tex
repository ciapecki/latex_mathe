%compile with pdflatex on papeeria

\documentclass[a4paper,12pt]{article}


\usepackage{fontawesome}

\usepackage{fancyhdr}
\usepackage{fancyheadings}
\usepackage[ngerman,german]{babel}
\usepackage{german}
\usepackage[utf8]{inputenc}
%\usepackage[latin1]{inputenc}
\usepackage[active]{srcltx}
%\usepackage{algorithm}
%\usepackage[noend]{algorithmic}
\usepackage{amsmath}
\usepackage{amssymb}
\usepackage{amsthm}
\usepackage{bbm}
\usepackage{enumerate}
\usepackage{graphicx}
\usepackage{ifthen}
\usepackage{listings}
\usepackage{enumitem}
%\usepackage{struktex}
\usepackage{hyperref}
\usepackage{tikz}
\usepackage{float}
\usepackage{subcaption}
\usepackage{array}
\captionsetup{compatibility=false}
\captionsetup[subfigure]{labelformat=empty}

\usepackage{pgfplots}
\pgfplotsset{compat=1.15}
\usepackage{mathrsfs}
\usetikzlibrary{arrows}

\definecolor{ccqqqq}{rgb}{0.8,0,0}
\definecolor{kolorwykresu}{rgb}{0.07,0.04,0.56}

\pagenumbering{gobble}

\usepackage{tabularray}
\usepackage{multirow}
\usepackage{booktabs,tabularx}

\DeclareMathSymbol{\shortminus}{\mathbin}{AMSa}{"39}

\renewcommand\tabularxcolumn[1]{m{#1}}% for vertical centering text in X column

\newcolumntype{L}[1]{>{\raggedright\let\newline\\\arraybackslash\hspace{0pt}}m{#1}}
\newcolumntype{C}[1]{>{\centering\let\newline\\\arraybackslash\hspace{0pt}}m{#1}}
\newcolumntype{R}[1]{>{\raggedleft\let\newline\\\arraybackslash\hspace{0pt}}m{#1}}

\newcolumntype{Y}{>{\centering\arraybackslash}X}

%%%%%%%%%%%%%%%%%%%%%%%%%%%%%%%%%%%%%%%%%%%%%%%%%%%%%%
%%%%%%%%%%%%%% EDIT THIS PART %%%%%%%%%%%%%%%%%%%%%%%%
%%%%%%%%%%%%%%%%%%%%%%%%%%%%%%%%%%%%%%%%%%%%%%%%%%%%%%
\newcommand{\Fach}{2. Schulaufgabe aus der Mathematik}
\newcommand{\Name}{}
\newcommand{\datum}{}
\newcommand{\Matrikelnummer}{}
\newcommand{\Semester}{7G}
\newcommand{\Uebungsblatt}{} %  <-- UPDATE ME
%%%%%%%%%%%%%%%%%%%%%%%%%%%%%%%%%%%%%%%%%%%%%%%%%%%%%%
%%%%%%%%%%%%%%%%%%%%%%%%%%%%%%%%%%%%%%%%%%%%%%%%%%%%%%

\setlength{\parindent}{0em}
\topmargin -1.0cm
\oddsidemargin 0cm
\evensidemargin 0cm
\setlength{\textheight}{9.2in}
\setlength{\textwidth}{6.0in}

%%%%%%%%%%%%%%%
%% Aufgaben-COMMAND
\newcommand{\Aufgabe}[1]{
  {
  \vspace*{0.5cm}
  \textsf{\textbf{Aufgabe #1}}
  \vspace*{0.2cm}
  
  }
}
%%%%%%%%%%%%%%
\hypersetup{
    pdftitle={\Fach{}: Übungsblatt \Uebungsblatt{}},
    pdfauthor={\Name},
    pdfborder={0 0 0}
}

\lstset{ %
language=java,
basicstyle=\footnotesize\tt,
showtabs=false,
tabsize=2,
captionpos=b,
breaklines=true,
extendedchars=true,
showstringspaces=false,
flexiblecolumns=true,
}

\newcommand*{\quadratbox}{\textbf{\fbox{\phantom{?}}}}%

\title{Übungsblatt \Uebungsblatt{}}
\author{\Name{}}

\begin{document}

\fancyhead{}
\fancyhead[C]{\includegraphics[height=2.5cm]{lukasLogo.png}
\vspace{2cm}
}

\thispagestyle{fancy}

\lhead{
%\vspace{1cm}
  \sf \LARGE \Fach{} %\small \Name{} - \Matrikelnummer{}
}
\rhead{\sf \Semester{}   \datum{}}

\vspace*{0.2cm}

\vspace{4cm}
Alle Lösungen müssen mit Nebenrechnungen und Begründungen nachvollziehbar sein!

%\rhead{\sf \Semester{} }
\vspace*{0.2cm}

%\begin{center}
%%\LARGE \sf \textbf{Übungsblatt \Uebungsblatt{}}
%\end{center}
%\vspace*{0.2cm}

%%%%%%%%%%%%%%%%%%%%%%%%%%%%%%%%%%%%%%%%%%%%%%%%%%%%%%
%% Insert your solutions here %%%%%%%%%%%%%%%%%%%%%%%%
%%%%%%%%%%%%%%%%%%%%%%%%%%%%%%%%%%%%%%%%%%%%%%%%%%%%%%

\vspace{1cm}
  Name: \underline{\hspace{7cm}}
  \hfill
  Datum: \underline{\hspace{4cm}}

\vspace{0.8cm}

\textbf{Hinweise:} Der Lösungsweg muss nachvollziehbar sein. Arbeitszeit \textbf{35 Minuten}.
%\vspace{0,5cm}Die Rechenwege müssen nachvollziehbar sein!
%
%\vspace{0,5cm} {TEIL A} - ohne Hilfsmittel - Bearbeitungszeit 30 Minuten
\vspace {0.8cm}
% 
%GEOMETRIE


\begin{center}
  \begin{tblr}{
      width=1\linewidth,
      colspec = {Q[c,6em]Q[c,4em]Q[c,4em]Q[c,4em]Q[c,4em]Q[c,6em]},
      rowspec = {Q[m]Q[m]Q[m]Q[m]Q[m]Q[m]},
      colsep = 0mm,
      %row{1} = {2em,azure2,fg=white,font=\large\bfseries\sffamily},
      row{1} = {2em,font=\large\bfseries\sffamily},
      hlines, vlines,
    }
    \textbf{Aufgabe} & \textbf{1} & \textbf{2} & \textbf{3} & 
    \textbf{4} & \textbf{Gesamt} \\
    {Mögliche \\ Punkte} & {7} & {18} & {10} & 12 & 47 \\
    {Erreichte \\ Punkte} &  &  &  &  &  \\
  \end{tblr}
\end{center}

\vspace{5cm}
%\centerline{\huge\bfseries\sffamily Viel Erfolg !!!}

\newpage


\Aufgabe{1:}
Vereinfache die Terme so weit wie möglich. Verwende dabei die binomischen
Formeln, wenn möglich.

\begin{enumerate}[label={\alph*)}]
  \item $(a-3)(a+3) -3(2a+5)^2=$
    \vspace{30mm}
  \item $(x-y)^2 + x^2-y^2 =$
    \vspace{30mm}
  \item $(0,2-k)(0,4g+3k)-0,2k(2k-5g)=$
    \vspace{30mm}
\end{enumerate}

\Aufgabe{2: }

Ersetze jeweils den/die Platzhalter, sodass äquivalente Terme enstehen.

\begin{enumerate}[label={\alph*)}]
  \item $\quadratbox x+4y=-2(3x-2y)$
  \item $\quadratbox a^2-\quadratbox ab = - \frac{a}{3}(2a+6b)$
  \item $\quadratbox ab + 7a^2b^2=-7ab(2-ab)$
  \item $\frac{15}{6}x-\frac{5}{3}y=\frac{5}{6}(\quadratbox x - \quadratbox y)$
\end{enumerate}

\newpage
\Aufgabe{3: (6BE + 4BE)}
\begin{enumerate}[label={\alph*)}]
  \item Konstruiere das Spiegelbild des Dreiecks $ABC$ mit $A(1|2)$ und $B(5|0)$ und $C(3|3)$, das an der Spiegelachse durch $D(-1|4)$ und $E(4|-1)$ gespiegelt wird. 
    Gib die Koordinaten der Punkte $A'$, $B'$ und $C'$ an.
\vspace{50mm}
\item Begründe, warum folgende Konstruktion einer Mittelsenkrechten falsch ist und \underline{beschreibe} wie man sie richtig konstruiert.
  \begin{figure}[H]
    %\vspace{-1cm}
    \centering
    \includegraphics[width=0.5\linewidth]{7G_2SA_image1.png}
  \end{figure}
\end{enumerate}

\newpage
\Aufgabe{4: }

\begin{minipage}[t]{0.55\textwidth}
  Herr Gärtner möchte seinen $a$ Meter langen und $b$ Meter breiten Garten mit Wegen versehen, wobei die Wege jeweils $0,6$ Meter breit sind. Er hat sich vier Möglichkeiten aufgezeichnet. Ordne diesen vier Möglichkeiten jeweils den richtigen Term zu, wenn die Größe der Blumenbeete $I$ bis $IV$ dargestellt werden soll (graue Bereiche).
\end{minipage}
\hspace*{0.75cm}
\begin{minipage}[t]{0.40\textwidth}
  \begin{figure}[H]
    \vspace{-1cm}
    \centering
    \includegraphics[width=1\linewidth]{7G_2SA_image2.png}
  \end{figure}
\end{minipage}

%\begin{enumerate}[label={\alph*)}]
\begin{enumerate}
  \item $(a-0,6)\cdot (b-1,8)$
  \item $(a-1,8)\cdot (b-0,6)$
  \item $(a-0,6)\cdot (b-0,6)$
  \item $(a-1,8)\cdot (b-0,6)$
  \item $(a-1,2)\cdot (b-1,2)$
  \item $(a-1,8)\cdot (b-1,2)$
\end{enumerate}

\vspace{2cm}

\Aufgabe{5: }
Ronja behauptet, dass die Terme $(3x - y)^2$ und $(y - 3x)^2$ äquivalent sind. Überprüfe diese Behauptung und begründe Deine Antwort.



\vspace{5cm}
\centerline{Viel Erfolg \faThumbsOUp }
\end{document}
