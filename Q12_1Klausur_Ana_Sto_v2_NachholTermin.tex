%compile with pdflatex on papeeria

\documentclass[a4paper,12pt]{article}
\usepackage{fancyhdr}
%\usepackage{fancyheadings}
\usepackage[ngerman,german]{babel}
\usepackage{german}
\usepackage[utf8]{inputenc}
%\usepackage[latin1]{inputenc}
\usepackage[active]{srcltx}
%\usepackage{algorithm}
%\usepackage[noend]{algorithmic}
\usepackage{amsmath}
\usepackage{amssymb}
\usepackage{amsthm}
\usepackage{bbm}
\usepackage{enumerate}
\usepackage{graphicx}
\usepackage{ifthen}
\usepackage{listings}
\usepackage{enumitem}
%\usepackage{struktex}
\usepackage{hyperref}
\usepackage{tikz}
\usepackage{float}
\usepackage{subcaption}
\usepackage{array}
\captionsetup{compatibility=false}
\captionsetup[subfigure]{labelformat=empty}

\usepackage{pgfplots}
\usepgfplotslibrary{fillbetween}
%\usetikzlibrary{patterns}
\pgfplotsset{compat=1.15}
\usepackage{mathrsfs}
\usetikzlibrary{arrows}

\pgfplotsset{grid style={dashed,gray}}

\definecolor{kolorwykresu}{rgb}{0.07,0.04,0.56}

\pagenumbering{gobble}

\usepackage{tabularray}
\usepackage{multirow}
\usepackage{booktabs,tabularx}
\renewcommand\tabularxcolumn[1]{m{#1}}% for vertical centering text in X column

\newcolumntype{L}[1]{>{\raggedright\let\newline\\\arraybackslash\hspace{0pt}}m{#1}}
\newcolumntype{C}[1]{>{\centering\let\newline\\\arraybackslash\hspace{0pt}}m{#1}}
\newcolumntype{R}[1]{>{\raggedleft\let\newline\\\arraybackslash\hspace{0pt}}m{#1}}

\newcolumntype{Y}{>{\centering\arraybackslash}X}

%%%%%%%%%%%%%%%%%%%%%%%%%%%%%%%%%%%%%%%%%%%%%%%%%%%%%%
%%%%%%%%%%%%%% EDIT THIS PART %%%%%%%%%%%%%%%%%%%%%%%%
%%%%%%%%%%%%%%%%%%%%%%%%%%%%%%%%%%%%%%%%%%%%%%%%%%%%%%
\newcommand{\Fach}{1. Klausur aus der Mathematik (A)}
\newcommand{\Name}{}
\newcommand{\datum}{}
\newcommand{\Matrikelnummer}{}
\newcommand{\Semester}{Q12/1}
\newcommand{\Uebungsblatt}{} %  <-- UPDATE ME
%%%%%%%%%%%%%%%%%%%%%%%%%%%%%%%%%%%%%%%%%%%%%%%%%%%%%%
%%%%%%%%%%%%%%%%%%%%%%%%%%%%%%%%%%%%%%%%%%%%%%%%%%%%%%

\setlength{\parindent}{0em}
\topmargin -1.0cm
\oddsidemargin 0cm
\evensidemargin 0cm
\setlength{\textheight}{10.2in}
\setlength{\textwidth}{6.0in}
%\setlength{\footskip}{-1cm}

%%%%%%%%%%%%%%%
%% Aufgaben-COMMAND
\newcommand{\Aufgabe}[1]{
  {
  \vspace*{0.5cm}
  \textsf{\textbf{Aufgabe #1}}
  \vspace*{0.2cm}
  
  }
}
%%%%%%%%%%%%%%
\hypersetup{
    pdftitle={\Fach{}: Übungsblatt \Uebungsblatt{}},
    pdfauthor={\Name},
    pdfborder={0 0 0}
}

\lstset{ %
language=java,
basicstyle=\footnotesize\tt,
showtabs=false,
tabsize=2,
captionpos=b,
breaklines=true,
extendedchars=true,
showstringspaces=false,
flexiblecolumns=true,
}

\title{Übungsblatt \Uebungsblatt{}}
\author{\Name{}}

\begin{document}

\fancyhead{}
\fancyhead[C]{\includegraphics[height=2.5cm]{lukasLogo.png}
\vspace{2cm}
}

\thispagestyle{fancy}
%\lhead{\sf \large \Fach{} \\ %\small \Name{} - \Matrikelnummer{}


\lhead{
%\vspace{1cm}
  \sf \LARGE \Fach{} %\small \Name{} - \Matrikelnummer{}
}
\rhead{\sf \Semester{}   \datum{}}

\vspace*{0.2cm}

\vspace{4cm}
Alle Lösungen müssen mit Nebenrechnungen und Begründungen nachvollziehbar sein!

%\rhead{\sf \Semester{} }
\vspace*{0.2cm}
%\begin{center}
%%\LARGE \sf \textbf{Übungsblatt \Uebungsblatt{}}
%\end{center}
%\vspace*{0.2cm}

%%%%%%%%%%%%%%%%%%%%%%%%%%%%%%%%%%%%%%%%%%%%%%%%%%%%%%
%% Insert your solutions here %%%%%%%%%%%%%%%%%%%%%%%%
%%%%%%%%%%%%%%%%%%%%%%%%%%%%%%%%%%%%%%%%%%%%%%%%%%%%%%

\vspace{1cm}
  Name: \underline{\hspace{7cm}}
%\draw[line width=1pt,color=ccqqqq,smooth,samples=100,domain=-6:7] plot(\x,{(1/12)*\x*\x+(1/3)*\x});
  \hfill
  Datum: \underline{\hspace{4cm}}

%\vspace{0,5cm}Die Rechenwege müssen nachvollziehbar sein!

%\vspace{1,5cm} {TEIL A} - ohne Hilfsmittel. Bearbeitungszeit 35 min.
\vspace {2cm}


\begin{center}
  \begin{tblr}{
      width=1\linewidth,
      colspec = {Q[c,6em]Q[c,4em]Q[c,4em]Q[c,4em]Q[c,4em]Q[c,4em]Q[c,4em]Q[c,6em]},
      rowspec = {Q[m]Q[m]Q[m]Q[m]Q[m]Q[m]Q[m]Q[m]},
      colsep = 0mm,
      %row{1} = {2em,azure2,fg=white,font=\large\bfseries\sffamily},
      row{1} = {2em,font=\large\bfseries\sffamily},
      hlines, vlines,
    }
    \textbf{Aufgabe} & \textbf{1} & \textbf{2} & \textbf{3} & 
    \textbf{4} & \textbf{5} & \textbf{6} & \textbf{Gesamt} \\
    {Mögliche \\ Punkte} & {7} & {8} & {13} & 5 & 7 & 10 & 50 \\
    {Erreichte \\ Punkte} &  &  &  &  &  &  &  \\
  \end{tblr}
\end{center}

\vspace{5cm}
\centerline{\huge\bfseries\sffamily Viel Erfolg !!!}

\newpage

\Aufgabe{1: (3BE+4BE)} 

\noindent
\begin{minipage}{0.6\textwidth}%\raggedleft
  Nebenstehende Abbildung zeigt den Graphen einer Funktion $f$.
  \begin{enumerate}[label={\alph*)}] 
    \item Begründen Sie, an welchen Stellen jede Stammfunktion von $f$ ein Minimum hat.
    \item Erläutern Sie, an welcher Stelle des Intervalls $[0;4]$ die Integralfunktion von $f$ zur unteren Grenze 0 ein lokales Maximum hat.
    \item Skizzieren Sie den Graphen einer Stammfunktion von $f$ in die Abbildung.
  \end{enumerate}
\end{minipage}
\hfill%
\begin{minipage}{0.3\textwidth}% adapt widths of minipages to your needs
  \includegraphics[width=\linewidth]{Q12_1Klausur_Ana_Sto_v2_NachholTermin_01.png}
\end{minipage}%


\Aufgabe{2: (3BE+5BE)} 
\begin{enumerate}[label={\alph*)}] 
  \item Berechnen Sie den Wert des Integrals:
    \[ \int\limits_{-1}^{0}\frac{6x+9}{3x+x^2-1}\,dx \]
  \item Bestimmen Sie den Inhalt der Fläche, die die Graphen der Funktionen ${f(x)=\sqrt{3x}}$ und $g(x) = 0,5x$ einschließen. (Skizze!)
\end{enumerate}

\Aufgabe{3: (2BE+2BE+4BE+2BE+3BE)} 
Ein Computervirus wird auf die Rechner dieser Welt losgelassen. Der Virus breitet sich erst aus und wird dann bekämpft. Die Funktion $v(t)=3t e^{-t}$  beschreibt modellhaft, wie viele Millionen Rechner vom Virus befallen sind. $t$ gibt die Zeit ab dem Ausbruch in Tagen an.
\begin{enumerate}[label={\alph*)}] 
  \item An welchem Tag waren die meisten Rechner betroffen und wie viele waren das?
  \item Berechnen Sie, an welchem Tag der Rückgang am stärksten war.
  \item Bestimmen Sie eine langfristige Prognose für diesen Virus und formulieren Sie eine Aussage dazu.
\end{enumerate}


\Aufgabe{4: (3BE+2BE)} 
\begin{enumerate}[label={\alph*)}] 
  \item Zeigen Sie: Für $a,b > 0$ und $c \in {\mathbb{R}}$  hat der Graph von $f(x)=ax^5-bx^3+cx$ drei Wendepunkte, die auf einer Geraden liegen.
\item Geben Sie einen Funktionsterm einer Funktion $f$ an, dessen Graph streng monoton steigt und rechtsgekrümmt ist.
\end{enumerate}


\Aufgabe{5: (7BE)} 
Sie organisieren ein Spiel mit einem zweifarbigen Glücksrad. Der Einsatz soll 5€ betragen und der Gewinner bekommt 100€. Als Bank wollen Sie unauffällig von dem Spiel profitieren und pro Spiel einen durchschnittlichen Gewinn von 10 Cent erzielen.\\
Berechnen Sie, wie groß der Gewinnsektor des Glücksrades sein muss, damit Ihre Rechnung aufgeht.

\Aufgabe{6: (4BE+3BE+3BE)} 
Bei einem Basketballtraining sollen die Teilnehmer Korbleger mit links üben. Sepps Trefferwahrscheinlichkeit liegt bei 0,4.
\begin{enumerate}[label={\alph*)}]
  \item Bestimmen Sie die Wahrscheinlichkeiten dafür, dass Sepp bei 5 Würfen (i) mindestens 2
    Körbe trifft, (ii) genau beim ersten und beim letzten Wurf trifft.
  \item Wie oft muss Sepp mindestens werfen, damit er mit einer Wahrscheinlichkeit von 98\%
    wenigstens einmal trifft?
\end{enumerate}




%\vspace{2cm}



\end{document}
