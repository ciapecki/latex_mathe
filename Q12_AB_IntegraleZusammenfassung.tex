%compile with pdflatex on papeeria

\documentclass[a4paper,12pt]{article}
\usepackage{fancyhdr}
\usepackage{fancyheadings}
\usepackage[ngerman,german]{babel}
\usepackage{german}
\usepackage[utf8]{inputenc}
%\usepackage[latin1]{inputenc}
\usepackage[active]{srcltx}
%\usepackage{algorithm}
%\usepackage[noend]{algorithmic}
\usepackage{amsmath}
\usepackage{amssymb}
\usepackage{amsthm}
\usepackage{bbm}
\usepackage{enumerate}
\usepackage{graphicx}
\usepackage{ifthen}
\usepackage{listings}
\usepackage{enumitem}
%\usepackage{struktex}
\usepackage{hyperref}
%\usepackage{tikz}

\pagenumbering{gobble}

%%%%%%%%%%%%%%%%%%%%%%%%%%%%%%%%%%%%%%%%%%%%%%%%%%%%%%
%%%%%%%%%%%%%% EDIT THIS PART %%%%%%%%%%%%%%%%%%%%%%%%
%%%%%%%%%%%%%%%%%%%%%%%%%%%%%%%%%%%%%%%%%%%%%%%%%%%%%%
\newcommand{\Fach}{aus der Mathematik (A)}
\newcommand{\Name}{}
\newcommand{\datum}{}
\newcommand{\Matrikelnummer}{}
\newcommand{\Semester}{Q12/1}
\newcommand{\Uebungsblatt}{} %  <-- UPDATE ME
%%%%%%%%%%%%%%%%%%%%%%%%%%%%%%%%%%%%%%%%%%%%%%%%%%%%%%
%%%%%%%%%%%%%%%%%%%%%%%%%%%%%%%%%%%%%%%%%%%%%%%%%%%%%%

\setlength{\parindent}{0em}
\topmargin -1.0cm
\oddsidemargin 0cm
\evensidemargin 0cm
\setlength{\textheight}{9.2in}
\setlength{\textwidth}{6.0in}

%%%%%%%%%%%%%%%
%% Aufgaben-COMMAND
\newcommand{\Aufgabe}[1]{
  {
  \vspace*{0.5cm}
  \textsf{\textbf{Aufgabe #1}}
  \vspace*{0.2cm}
  
  }
}
%%%%%%%%%%%%%%
\hypersetup{
    pdftitle={\Fach{}: Übungsblatt \Uebungsblatt{}},
    pdfauthor={\Name},
    pdfborder={0 0 0}
}

\lstset{ %
language=java,
basicstyle=\footnotesize\tt,
showtabs=false,
tabsize=2,
captionpos=b,
breaklines=true,
extendedchars=true,
showstringspaces=false,
flexiblecolumns=true,
}

%\title{Übungsblatt \Uebungsblatt{}}
\author{\Name{}}

\begin{document}
%\thispagestyle{fancy}
%\lhead{\sf \large \Fach{} \\ %\small \Name{} - \Matrikelnummer{}
%}
%\rhead{\sf \Semester{} \\  \datum{}}
%\vspace*{0.2cm}
%\begin{center}
%%\LARGE \sf \textbf{Übungsblatt \Uebungsblatt{}}
%\end{center}
%\vspace*{0.2cm}

%%%%%%%%%%%%%%%%%%%%%%%%%%%%%%%%%%%%%%%%%%%%%%%%%%%%%%
%% Insert your solutions here %%%%%%%%%%%%%%%%%%%%%%%%
%%%%%%%%%%%%%%%%%%%%%%%%%%%%%%%%%%%%%%%%%%%%%%%%%%%%%%

  Name: \underline{\hspace{7cm}}
  \hfill
  Datum: \underline{\hspace{4cm}}

\vspace{2cm}

%\begin{tikzpicture}[scale=0.5]
%    % Draw a line at 30 degrees and of length 3
%    %\draw (0,0) -- (0:2.5cm);
%  \draw[black] (0,0) -- (0,6) -- (5,9) -- (10,6);
%  \draw[black] (5,9) -- (12,11) -- (14,10);
%  \draw[black] (0,0) -- (10,0) -- (10,6);
%  \draw[black,thin,dashed]  (0,8) -- (4,3);
%  \draw[black,thin,dashed]  (2,3) -- (2,5);
%  \draw[red,dashed] (0,0) -- (3,1.5) -- (8,1.5);
%  \draw[red] (4,0) -- (5,0.5) -- (6.5,0.5) -- (8,1.5);
%\end{tikzpicture}


%{\LARGE Zusammenfassung der wesentlichen Ableitungsregeln}
%\\

\fbox{%
  \begin{minipage}{\textwidth}
  {\large Grundintegrale:}\\
    %\begin{equation}
      \[ \int x^r dx = \frac{1}{r+1}x^{r+1}+c \quad (r \neq -1)\]
      \[  \int \sin x dx = -\cos{x}+c\]
      \[  \int e^x dx = e^x+c\]
      \[  \int \frac{1}{x}dx = \ln |x|+c\]
      \[  \int \cos{x}\ dx = \sin{x}+c\]
      \[  \int \ln{x}\ dx = x\cdot \ln{x}-x+c\]
    %\end{equation}
  \end{minipage}
}

\vspace{1cm}

\fbox{%
  \begin{minipage}{\textwidth}
  {\large Faktorregel:}\\
    Einen konstanten Faktor darf man vor das Integral ziehen.\\
      \[ \int k\cdot f(x)\,dx = k\cdot \int f(x)\,dx \]
  {\large Summenregel:}\\
    Eine Summe darf man gliederweise integrieren.
    \[ \int (f(x) + g(x))\,dx = \int f(x)\ dx + \int g(x)\,dx \]
  \end{minipage}
}

\vspace{1cm}

\fbox{%
  \begin{minipage}{\textwidth}
    Es sei $F$ eine Stammfunktion von $f$ und $a\neq0$. Dann gilt:
      \[ \int f(ax+b)\,dx = \frac{1}{a}F(ax+b)+c \]
  \end{minipage}
}

\vspace{1cm}
\fbox{%
  \begin{minipage}{\textwidth}
    \[ \int \frac{f'(x)}{f(x)}\,dx = \ln{|f(x)|}+c \quad \quad \quad
       \int f'(x)\cdot e^{f(x)}\,dx = e^{f(x)}+c \]
  \end{minipage}
}


\newpage

\Aufgabe{1: Grundwissen: ,,Aufleiten''}
Bestimme zu $f$ eine Stammfunktion $F$, d.h. eine Funktion, deren Ableitung $f(x)$ ist.
\begin{enumerate}[label={\alph*)}, topsep=5pt,itemsep=4ex,partopsep=1ex,parsep=1ex]
  \item $f(x)=x^3-3x^2$
  \item $f(x)=7x^7-7$
  \item $f(x)=x^2-a^2$
  \item $f(r)=4\pi r^2$
  \item $f(x)=\frac{1}{x^2}$
  \item $f(x)=\frac{1}{x^3} - \frac{1}{x^4}$
  \item $f(x)=2x+\frac{1}{2x^2}$
  \item $f(x)=\frac{1}{(x+2)^2}$
  \item $f(x)=(x+\sqrt{2})^3$
  \item $f(x)=(2x+3)^4$
  \item $f(x)=(3x-4)^5$
  \item $f(x)=\frac{1}{(2x+1)^3}$
  \item $f(x)=\sqrt{x}$
  \item $f(x)=\frac{1}{\sqrt{x}}$
  \item $f(x)=\sqrt{x-3}$
  \item $f(x)=\sqrt{2x-3}$
  \item $f(x)=\sin x$
  \item $f(x)=\cos(x+2)$
  \item $f(x)=\sin{2x}$
  \item $f(x)=\cos(\frac{\pi}{2}-x)$
\end{enumerate}

\vspace{1cm}
\Aufgabe{2: Anwendungen der Grundintegrale}
Berechne das bestimmte Integral. Deute das Ergebnis am Graphen von $f$.
\begin{enumerate}[label={\alph*)}, topsep=5pt,itemsep=4ex,partopsep=1ex,parsep=1ex]
  \item $\int_{-2}^{2} x^4\,dx$
  \item $\int_{-2}^{2} 6x^5\,dx$
  \item $\int_{-3}^{4}\,dx$
  \item $\int_{-3}^{4}4\sqrt{2}\,dx$
  \item $\int_{-3}^{4}0\,dx$
  \item $\int_{1}^{2}\frac{1}{x^2}\,dx$
  \item $\int_{-2}^{-1}\frac{1}{x^2}\,dx$
  \item $\int_{1}^{2}\frac{1}{x}\,dx$
  \item $\int_{0,5}^{1}\frac{1}{x}\,dx$
  \item $\int_{-2}^{-1}\frac{1}{x}\,dx$
  \item $\int_{0}^{9}\sqrt{x}\,dx$
  \item $\int_{1}^{9}\frac{1}{\sqrt{x}}\,dx$
  \item $\int_{0}^{9}x\sqrt{x}\,dx$
  \item $\int_{0}^{\frac{\pi}{2}}\sin{x}\,dx$
  \item $\int_{0}^{\frac{\pi}{2}}\cos{x}\,dx$
  \item $\int_{1}^{2}\ln{x}\,dx$
  \item $\int_{0,5}^{1}\ln{x}\,dx$
  \item $\int_{0,5}^{2}\ln{x}\,dx$
  \item $\int_{-1}^{0}e^x\,dx$
  \item $\int_{-1000}^{0}e^x\,dx$
\end{enumerate}


\vspace{1cm}
\Aufgabe{3: Integration linear transformierter Grundfunktionen}
Bestimme:
\begin{enumerate}[label={\alph*)}, topsep=5pt,itemsep=4ex,partopsep=1ex,parsep=1ex]
  \item $\int 5\cdot (x-3)^4\,dx$
  \item $\int (3x+4)^5\,dx$
  \item $\int (\frac{1}{2}x-\frac{1}{3})^3\,dx$
  \item $\int 6\cdot (4-3x)^5\,dx$
  \item $\int \frac{1}{(2x-3)^2}\,dx$
  \item $\int \frac{1}{(3-2x)^3}\,dx$
  \item $\int \frac{1}{2x+1}\,dx$
  \item $\int \frac{8}{3-4x}\,dx$
  \item $\int \sqrt{2x+3}\,dx$
  \item $\int (\sqrt{2}x+\sqrt{3})\,dx$
  \item $\int \frac{3}{\sqrt{3x-4}}\,dx$
  \item $\int \frac{3}{\sqrt{3}x-4}\,dx$
  \item $\int 2\cdot \sin(-t)\,dt$
  \item $\int 2\cdot \cos(\pi-2t)\,dt$
  \item $\int \sqrt{2}\cdot \cos(-\frac{1}{2}t)\,dt$
  \item $\int e^{-2x}\,dx$
  \item $\int e^{\frac{1}{2}x-1}\,dx$
  \item $\int \frac{e^x+e^{-x}}{2}\,dx$
  \item $\int (e^x+e^{-x})^2\,dx$
  \item $\int \ln(2x)\,dx$
  \item $\int \ln(2-x)\,dx$
\end{enumerate}



\vspace{1cm}
\Aufgabe{4: Logarithmische Integration}
Bestimme:
\begin{enumerate}[label={\alph*)}, topsep=5pt,itemsep=4ex,partopsep=1ex,parsep=1ex]
  \item $\int \frac{2x}{x^2+2}\,dx$
  \item $\int \frac{2x+1}{x^2+x+1}\,dx$
  \item $\int \frac{x^2}{x^3-2}\,dx$
  \item $\int \frac{1}{5x+4}\,dx$
  \item $\int \frac{3x-1}{3x^2-2x+4}\,dx$
  \item $\int \frac{e^x+1}{e^x+x}\,dx$
  \item $\int \frac{e^{2x}}{e^{2x}+e}\,dx$
  \item $\int \frac{e^x-e^{-x}}{e^x+e^{-x}}\,dx$
  \item $\int \frac{\sin{x}}{\cos{x}}\,dx$
\end{enumerate}

\Aufgabe{5: Ableitung des Exponenten gesucht!}
Bestimme:
\begin{enumerate}[label={\alph*)}, topsep=5pt,itemsep=4ex,partopsep=1ex,parsep=1ex]
  \item $\int 4\cdot e^{2x}\,dx$
  \item $\int 4x\cdot e^{x^2}\,dx$
  \item $\int e^{1-4x}\,dx$
  \item $\int x\cdot e^{1-4x^2}\,dx$
  \item $\int \frac{e^{1+2\ln{x}}}{x}\,dx$
  \item $\int \frac{e^{\sqrt{x}}}{\sqrt{x}}\,dx$
\end{enumerate}

%%%%%%%%%%%%%%%%%%%%%%%%%%%%%%%%%%%%%%%%%%%%%%%%%%%%%%
%%%%%%%%%%%%%%%%%%%%%%%%%%%%%%%%%%%%%%%%%%%%%%%%%%%%%%
\end{document}
