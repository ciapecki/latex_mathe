%compile with pdflatex on papeeria

\documentclass[a4paper,12pt]{article}
\usepackage{fancyhdr}
%\usepackage{fancyheadings}
\usepackage[ngerman,german]{babel}
\usepackage{german}
\usepackage[utf8]{inputenc}
%\usepackage[latin1]{inputenc}
\usepackage[active]{srcltx}
%\usepackage{algorithm}
%\usepackage[noend]{algorithmic}
\usepackage{amsmath}
\usepackage{amssymb}
\usepackage{amsthm}
\usepackage{bbm}
\usepackage{enumerate}
\usepackage{graphicx}
\usepackage{ifthen}
\usepackage{listings}
\usepackage{enumitem}
%\usepackage{struktex}
\usepackage{hyperref}
\usepackage{tikz}
\usepackage{float}
\usepackage{subcaption}
\usepackage{array}
\captionsetup{compatibility=false}
\captionsetup[subfigure]{labelformat=empty}

\usepackage{pgfplots}
\usepgfplotslibrary{fillbetween}
%\usetikzlibrary{patterns}
\pgfplotsset{compat=1.15}
\usepackage{mathrsfs}
\usetikzlibrary{arrows}

\pgfplotsset{grid style={dashed,gray}}

\definecolor{kolorwykresu}{rgb}{0.07,0.04,0.56}

\pagenumbering{gobble}

\usepackage{tabularray}
\usepackage{multirow}
\usepackage{booktabs,tabularx}
\renewcommand\tabularxcolumn[1]{m{#1}}% for vertical centering text in X column

\newcolumntype{L}[1]{>{\raggedright\let\newline\\\arraybackslash\hspace{0pt}}m{#1}}
\newcolumntype{C}[1]{>{\centering\let\newline\\\arraybackslash\hspace{0pt}}m{#1}}
\newcolumntype{R}[1]{>{\raggedleft\let\newline\\\arraybackslash\hspace{0pt}}m{#1}}

\newcolumntype{Y}{>{\centering\arraybackslash}X}

%%%%%%%%%%%%%%%%%%%%%%%%%%%%%%%%%%%%%%%%%%%%%%%%%%%%%%
%%%%%%%%%%%%%% EDIT THIS PART %%%%%%%%%%%%%%%%%%%%%%%%
%%%%%%%%%%%%%%%%%%%%%%%%%%%%%%%%%%%%%%%%%%%%%%%%%%%%%%
\newcommand{\Fach}{1. Klausur aus der Mathematik (A)}
\newcommand{\Name}{}
\newcommand{\datum}{}
\newcommand{\Matrikelnummer}{}
\newcommand{\Semester}{Q12/1}
\newcommand{\Uebungsblatt}{} %  <-- UPDATE ME
%%%%%%%%%%%%%%%%%%%%%%%%%%%%%%%%%%%%%%%%%%%%%%%%%%%%%%
%%%%%%%%%%%%%%%%%%%%%%%%%%%%%%%%%%%%%%%%%%%%%%%%%%%%%%

\setlength{\parindent}{0em}
\topmargin -1.0cm
\oddsidemargin 0cm
\evensidemargin 0cm
\setlength{\textheight}{10.2in}
\setlength{\textwidth}{6.0in}
%\setlength{\footskip}{-1cm}

%%%%%%%%%%%%%%%
%% Aufgaben-COMMAND
\newcommand{\Aufgabe}[1]{
  {
  \vspace*{0.5cm}
  \textsf{\textbf{Aufgabe #1}}
  \vspace*{0.2cm}
  
  }
}
%%%%%%%%%%%%%%
\hypersetup{
    pdftitle={\Fach{}: Übungsblatt \Uebungsblatt{}},
    pdfauthor={\Name},
    pdfborder={0 0 0}
}

\lstset{ %
language=java,
basicstyle=\footnotesize\tt,
showtabs=false,
tabsize=2,
captionpos=b,
breaklines=true,
extendedchars=true,
showstringspaces=false,
flexiblecolumns=true,
}

\title{Übungsblatt \Uebungsblatt{}}
\author{\Name{}}

\begin{document}

\fancyhead{}
\fancyhead[C]{\includegraphics[height=2.5cm]{lukasLogo.png}
\vspace{2cm}
}

\thispagestyle{fancy}
%\lhead{\sf \large \Fach{} \\ %\small \Name{} - \Matrikelnummer{}


\lhead{
%\vspace{1cm}
  \sf \LARGE \Fach{} %\small \Name{} - \Matrikelnummer{}
}
\rhead{\sf \Semester{}   \datum{}}

\vspace*{0.2cm}

\vspace{4cm}
Alle Lösungen müssen mit Nebenrechnungen und Begründungen nachvollziehbar sein!

%\rhead{\sf \Semester{} }
\vspace*{0.2cm}
%\begin{center}
%%\LARGE \sf \textbf{Übungsblatt \Uebungsblatt{}}
%\end{center}
%\vspace*{0.2cm}

%%%%%%%%%%%%%%%%%%%%%%%%%%%%%%%%%%%%%%%%%%%%%%%%%%%%%%
%% Insert your solutions here %%%%%%%%%%%%%%%%%%%%%%%%
%%%%%%%%%%%%%%%%%%%%%%%%%%%%%%%%%%%%%%%%%%%%%%%%%%%%%%

\vspace{1cm}
  Name: \underline{\hspace{7cm}}
%\draw[line width=1pt,color=ccqqqq,smooth,samples=100,domain=-6:7] plot(\x,{(1/12)*\x*\x+(1/3)*\x});
  \hfill
  Datum: \underline{\hspace{4cm}}

%\vspace{0,5cm}Die Rechenwege müssen nachvollziehbar sein!

%\vspace{1,5cm} {TEIL A} - ohne Hilfsmittel. Bearbeitungszeit 35 min.
\vspace {2cm}


\begin{center}
  \begin{tblr}{
      width=1\linewidth,
      colspec = {Q[c,6em]Q[c,4em]Q[c,4em]Q[c,4em]Q[c,4em]Q[c,4em]Q[c,4em]Q[c,6em]},
      rowspec = {Q[m]Q[m]Q[m]Q[m]Q[m]Q[m]Q[m]Q[m]},
      colsep = 0mm,
      %row{1} = {2em,azure2,fg=white,font=\large\bfseries\sffamily},
      row{1} = {2em,font=\large\bfseries\sffamily},
      hlines, vlines,
    }
    \textbf{Aufgabe} & \textbf{1} & \textbf{2} & \textbf{3} & 
    \textbf{4} & \textbf{5} & \textbf{6} & \textbf{Gesamt} \\
    {Mögliche \\ Punkte} & {7} & {8} & {13} & 5 & 7 & 10 & 50 \\
    {Erreichte \\ Punkte} &  &  &  &  &  &  &  \\
  \end{tblr}
\end{center}

\vspace{5cm}
\centerline{\huge\bfseries\sffamily Viel Erfolg !!!}

\newpage

\Aufgabe{1: (3BE+4BE)} 
Gegeben ist der Graph der Funktion $f$. 
\begin{enumerate}[label={\alph*)}] 
  \item Bestimmen Sie näherungsweise den Wert des Integrals ${\int_{0}^{10} f(x)\, dx}$ und ermitteln Sie die prozentuale Abweichung von dem realen Wert von 51.
\begin{figure}[H]
  \centering
  \includegraphics[width=0.8\columnwidth]{211117_8.png}
  %\caption{A boat.}
  %\label{fig:boat1}
\end{figure}
%\newpage
  \item Gegeben ist der Graph der Funktion $f$ aus Teilaufgabe a). Skizzieren Sie einen möglichen Verlauf der Ableitungs- und Stammfunktion von $f$ in das gegebene Koordinatensystem.
\end{enumerate}
\begin{figure}[H]
  \centering
  \includegraphics[width=0.8\columnwidth]{211117_8.png}
  %\caption{A boat.}
  %\label{fig:boat1}
\end{figure}


\vspace{3cm}

\Aufgabe{2: (3BE+5BE)} 
\begin{enumerate}[label={\alph*)}] 
  \item Berechnen Sie den Wert des Integrals:
    \[ \int_{3}^{5} \frac{x+1}{2(0,5x^2+x)}\,dx \]
  \item Bestimmen Sie den Inhalt der Fläche, die die Graphen der Funktionen ${f(x)= \frac{1}{4}x^2}$ und ${g(x) = x+3}$ mit der $y$-Achse im 1. Quadranten einschließen. (Skizze!)
\end{enumerate}

\Aufgabe{3: (2BE+2BE+4BE+2BE+3BE)} 
Ein Antibiotikum wird in unterschiedlichen Wirkstoffkonzentrationen produziert. Den zeitlichen Verlauf der Wirkstoffkonzentration im Blut beschreibt ein Mathematiker durch folgende Funktionenschar:
\[f_k: t \rightarrow  k \cdot t \cdot e^{-0,2t} \quad  t\ge0 \quad k>0 \]

Es wird die Zeit $t$ in Stunden seit der Einnahme und die Wirkstoffkonzentration $f(t)$ im Blut in $\frac{mg}{l}$ gemessen.

\begin{enumerate}[label={\alph*)}]
  \item Bestimmen Sie $f_k(0)$ sowie $\lim \limits_{t \to \infty} f_k(t)$ und interpretieren Sie im Sachzusammenhang.
  \item Zeigen Sie, dass für den Term der Ableitungsfunktion von $f_k$, gilt: 
    \[g_k (t) = k \cdot e^{-0,2t} (1-0,2t) \]
  \item Bestimmen Sie rechnerisch Monotonie und Extremwerte der Funktionenschar $f_k$ nach Art und Lage. 
    Formulieren Sie nun den Einfluss des Parameters $k$ in Worten. Argumentieren Sie dabei unter Einbeziehung der Anwendungssituation.
  \item Die maximale Wirkstoffkonzentration im Blut soll $11\frac{mg}{l}$ betragen. Ermitteln Sie den zugehörigen Parameter $k$.
%\end{enumerate}

Gerundetes Zwischenergebnis für die folgenden Aufgaben: 
\[ f(t)=6 \cdot t \cdot e^{-0,2t} \]


  \item Bestimmen Sie den Zeitpunkt, zu dem der Wirkstoff am stärksten abgebaut wird. Argumentieren Sie unter Verwendung der mathematischen Fachsprache und dokumentieren Sie Ihren Gedankengang ausführlich im Anwendungskontext.
\end{enumerate}

\newpage
\Aufgabe{4: (3BE+2BE)} 
\begin{enumerate}[label={\alph*)}] 
  \item Bestimmen Sie eine ganzrationale Funktion höchstens vierten Grades, deren Graph bzgl. der $y$-Achse symmetrisch ist und in $P (-1|1)$ eine Wendetangente mit der Steigung 3 hat.
  \item Geben Sie einen Funktionsterm einer Funktion $f$ an, dessen Graph streng monoton steigt und rechtsgekrümmt ist.
\end{enumerate}


%\Aufgabe{5: (5)} 
%In einer Urne sind Kugeln mit den Buchstaben: $AGKMOLNES$
%\begin{enumerate}[label={\alph*)}] 
%  \item Es wird mit Zurücklegen gezogen. Mit welcher Wahrscheinlichkeit wird das Wort $KAMEL$ in dieser Reihenfolge gezogen?
%  \item Es wird ohne Zurücklegen gezogen. Mit welcher Wahrscheinlichkeit kommt jetzt das Wort $KAMEL$ in dieser Reihenfolge?
%  \item Welche Wahrscheinlichkeit erhält man jeweils wenn das Wort bei a) und b) $KAMELE$ ist?
%\end{enumerate}


\Aufgabe{5: (7BE)} 
Bei einem Basketballtraining sollen die Teilnehmer Korbleger mit links üben. Sepps Trefferwahrscheinlichkeit liegt bei 0,4.
\begin{enumerate}[label={\alph*)}] 
  \item Bestimmen Sie die Wahrscheinlichkeiten dafür, dass Sepp bei 5 Würfen\\
    i) mindestens 2 Körbe trifft,  ii) genau beim ersten und beim letzten Wurf trifft.
  \item Wie oft muss Sepp mindestens werfen, damit er mit einer Wahrscheinlichkeit von 98\% wenigstens einmal trifft?
\end{enumerate}

\Aufgabe{6: (4BE+3BE+3BE)} 
Mit den beiden Glücksrädern wird Ihnen folgendes Spiel angeboten: Sie legen 5€ auf den Tisch. Dann wird eines der Glücksräder zweimal gedreht. Nach jeder Drehung geschieht Folgendes: Zeigt der Pfeil auf $\cdot$2, so wird der gerade auf dem Tisch liegende Betrag verdoppelt, zeigt der Pfeil auf $\cdot$0.5, wird er halbiert.

\begin{figure}[H]
  \centering
  \includegraphics[width=0.7\columnwidth]{211203_gluecksraeder.png}
\end{figure}

\begin{enumerate}[label={\alph*)}] 
  \item Berechnen Sie den Erwartungswert für Ihren Gewinn, wenn das Glücksrad 1 benutzt wird.
  \item Berechnen Sie den Erwartungswert für Ihren Gewinn in Abhängigkeit von der Wahrscheinlichkeit $p$, mit der der Pfeil auf dem Feld $\cdot$2 stehen bleibt, wenn das Glücksrad 2 benutzt wird.
  \item Konstruieren Sie ein Glücksrad ähnlich wie das Glücksrad 2, sodass das Spiel als fair bezeichnet werden kann. Wie groß wird der Mittelpunktwinkel für das Feld $\cdot$2?
\end{enumerate}
%\vspace{2cm}



\end{document}
