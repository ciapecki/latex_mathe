\documentclass[a4paper,12pt]{article}


\usepackage{fontawesome}
\usepackage{fancyhdr}
\usepackage{fancyheadings}
\usepackage[ngerman,german]{babel}
\usepackage{german}
\usepackage[utf8]{inputenc}
%\usepackage[latin1]{inputenc}
\usepackage[active]{srcltx}
%\usepackage{svg}
%\usepackage{algorithm}
%\usepackage[noend]{algorithmic}
\usepackage{eurosym}
\usepackage{amsmath}
\usepackage{amssymb}
\usepackage{amsthm}
\usepackage{bbm}
\usepackage{enumerate}
\usepackage{graphicx}
\usepackage{ifthen}
\usepackage{listings}
\usepackage{enumitem}
%\usepackage{struktex}
\usepackage{hyperref}
\usepackage{tikz}
\usepackage{float}
\usepackage{subcaption}

%\makeatletter
%\@ifundefined{c@rownum}{}{\let\c@rownum\relax}
%\makeatother
%\usepackage[table]{xcolor} % For coloring table cells
%\usepackage{colortbl}
\usepackage{array}
\captionsetup{compatibility=false}
\captionsetup[subfigure]{labelformat=empty}

\usepackage{pgfplots}
\pgfplotsset{compat=1.18}
\usepackage{mathrsfs}
\usetikzlibrary{arrows}

\definecolor{ccqqqq}{rgb}{0.8,0,0}
\definecolor{kolorwykresu}{rgb}{0.07,0.04,0.56}

\pagenumbering{gobble}

\usepackage{tabularray}
\usepackage{multirow}
\usepackage{booktabs,tabularx}

%\DeclareMathSymbol{\shortminus}{\mathbin}{AMSa}{"39}

\renewcommand\tabularxcolumn[1]{m{#1}}% for vertical centering text in X column

\newcolumntype{L}[1]{>{\raggedright\let\newline\\\arraybackslash\hspace{0pt}}m{#1}}
\newcolumntype{C}[1]{>{\centering\let\newline\\\arraybackslash\hspace{0pt}}m{#1}}
\newcolumntype{R}[1]{>{\raggedleft\let\newline\\\arraybackslash\hspace{0pt}}m{#1}}

\newcolumntype{Y}{>{\centering\arraybackslash}X}

%%%%%%%%%%%%%%%%%%%%%%%%%%%%%%%%%%%%%%%%%%%%%%%%%%%%%%
%%%%%%%%%%%%%% EDIT THIS PART %%%%%%%%%%%%%%%%%%%%%%%%
%%%%%%%%%%%%%%%%%%%%%%%%%%%%%%%%%%%%%%%%%%%%%%%%%%%%%%
\newcommand{\Fach}{1. Schulaufgabe aus der Mathematik}
\newcommand{\Name}{}
\newcommand{\datum}{}
\newcommand{\Matrikelnummer}{}
\newcommand{\Semester}{11g}
\newcommand{\Uebungsblatt}{} %  <-- UPDATE ME
%%%%%%%%%%%%%%%%%%%%%%%%%%%%%%%%%%%%%%%%%%%%%%%%%%%%%%
%%%%%%%%%%%%%%%%%%%%%%%%%%%%%%%%%%%%%%%%%%%%%%%%%%%%%%

\setlength{\parindent}{0em}
\topmargin -1.0cm
\oddsidemargin 0cm
\evensidemargin 0cm
\setlength{\textheight}{9.2in}
\setlength{\textwidth}{6.0in}

%%%%%%%%%%%%%%%
%% Aufgaben-COMMAND
\newcommand{\Aufgabe}[1]{
  {
  \vspace*{0.5cm}
  \textsf{\textbf{Aufgabe #1}}
  \vspace*{0.2cm}
  
  }
}
%%%%%%%%%%%%%%
\hypersetup{
    pdftitle={\Fach{}: Übungsblatt \Uebungsblatt{}},
    pdfauthor={\Name},
    pdfborder={0 0 0}
}

\lstset{ %
language=java,
basicstyle=\footnotesize\tt,
showtabs=false,
tabsize=2,
captionpos=b,
breaklines=true,
extendedchars=true,
showstringspaces=false,
flexiblecolumns=true,
}

\newcommand*{\quadratbox}{\textbf{\fbox{\phantom{\huge{?}}}}}%

\title{Übungsblatt \Uebungsblatt{}}
\author{\Name{}}

\begin{document}

\fancyhead{}
%\fancyhead[C]{\includegraphics[height=1.5cm]{lukasLogoV2.png}
\vspace{1cm}
%}

\thispagestyle{fancy}

\lhead{
%\vspace{1cm}
  \sf \LARGE \Fach{} %\small \Name{} - \Matrikelnummer{}
}
\rhead{\sf \Semester{}   \datum{}}


\vspace*{-0.2cm}

%\vspace{2cm}
Alle Lösungen müssen mit Nebenrechnungen und Begründungen nachvollziehbar sein!

%\rhead{\sf \Semester{} }
%\vspace*{0.2cm}

%\begin{center}
%%\LARGE \sf \textbf{Übungsblatt \Uebungsblatt{}}
%\end{center}
%\vspace*{0.2cm}

%%%%%%%%%%%%%%%%%%%%%%%%%%%%%%%%%%%%%%%%%%%%%%%%%%%%%%
%% Insert your solutions here %%%%%%%%%%%%%%%%%%%%%%%%
%%%%%%%%%%%%%%%%%%%%%%%%%%%%%%%%%%%%%%%%%%%%%%%%%%%%%%

\vspace{1cm}
  Name: \underline{\hspace{7cm}}
  \hfill
  Datum: \underline{\hspace{4cm}}

%\vspace{0.8cm}

%\textbf{Hinweise:} Der Lösungsweg muss nachvollziehbar sein. Arbeitszeit \textbf{45 Minuten}.
%\vspace{0,5cm}Die Rechenwege müssen nachvollziehbar sein!
%
%\vspace{0,5cm} {TEIL A} - ohne Hilfsmittel - Bearbeitungszeit 30 Minuten
%\vspace {0.8cm}
% 
%GEOMETRIE

%\begin{center}
%  \begin{tblr}{
%      width=1\linewidth,
%      colspec = {Q[c,5em]Q[c,2em]Q[c,2em]Q[c,2em]Q[c,2em]Q[c,2em]Q[c,2em]Q[c,2em]Q[c,2em]Q[c,2em]Q[c,2em]Q[c,5em]Q[c,3em]},
%      rowspec = {Q[m]Q[m]Q[m]Q[m]Q[m]Q[m]Q[m]Q[m]Q[m]Q[m]Q[m]Q[m]Q[m]},
%      colsep = 0mm,
%      %row{1} = {2em,azure2,fg=white,font=\large\bfseries\sffamily},
%      row{1} = {2em,font=\large\bfseries\sffamily},
%      hlines, vlines,
%    }
%    \textbf{Aufgabe} & \textbf{1} & \textbf{2} & \textbf{3} & 
%    \textbf{4} & \textbf{5} & \textbf{6}& \textbf{7}& \textbf{8}& \textbf{9}& \textbf{10}&
%    \textbf{Gesamt} & \textbf{Note}\\
%    {Mögliche \\ Punkte} & 4 & 4 & 4 & 8 & 13 & 13 & 3 & 6 & 7 & 3 & 65 \\
%    {Erreichte \\ Punkte} &  &  &  &  & & & & & & & & \\
%  \end{tblr}
%\end{center}

%\vspace{0.3cm}
%\newpage\null\thispagestyle{empty}\newpage
%\newpage
%\vspace{1,5cm} {TEIL A} - ohne Hilfsmittel. Bearbeitungszeit 25 min.
%\vspace {0,2cm}

\Aufgabe{1: (7BE+4BE)}
In folgendem Koordinatensystem sind die Graphen der Funktionen $f$ und $f^*$ abgebildet.\\

\begin{tikzpicture}[scale=2]
  \begin{axis}[
    axis lines=middle,
    xmin=-7, xmax=3, ymin=-3, ymax=5, % Further reduced axis limits
    xlabel={$x$}, ylabel={$y$},
    xtick={-7,-6,-5,-4,-3,-2,-1,0,1,2,3}, % Explicit tick values for x-axis
    ytick={-3,-2,-1,0,1,2,3,4,5,6}, % Explicit tick values for y-axis
    tick label style={font=\tiny},
    xticklabel style={anchor=north, yshift=-0.5ex},
    yticklabel style={anchor=east, xshift=-0.5ex},
    legend style={font=\tiny,legend pos=outer north east},
    axis equal,
    grid=both, % Enable grid for both axes
    grid style={line width=0.1mm, draw=gray!30}, % Customize grid line style
    major grid style={line width=0.1mm,draw=gray!30}, % Style for major grid
    minor grid style={dashed, draw=gray!30} % Style for minor grid
    %width=10cm, % Set explicit width
    %height=8cm, % Set explicit height
    %scaled ticks=base 10:3, % Scaled ticks for better formatting
    enlarge x limits=0.15, % Add margin to x-axis
    enlarge y limits=0.15, % Add margin to y-axis
    clip=true, % Clip the plot to axis boundaries
    ]
    % Plot the function with domain limited to avoid large values
    \addplot+[no marks, domain=-2:2, samples=100, smooth] {x^3};
    \addplot+[no marks, domain=-8:3, samples=100, smooth] {(-(1/3)*(x+3))^3 + 2};
    % Add "0" at the origin
    \node[anchor=north east, font=\tiny] at (axis cs:0,0) {0};

    \node[anchor=north west, font=\large, color=blue] at (axis cs:-3,-3) {$G_f$};
    \node[anchor=north west, font=\large, color=red] at (axis cs:-6,5) {$G_{f^*}$};
  \end{axis}
\end{tikzpicture}


\begin{enumerate}[label={\alph*)}, topsep=5pt,itemsep=4ex,partopsep=1ex,parsep=1ex]
  \item Beschreibe in der richtigen Reihenfolge zunächst in Stichpunkten, durch welche Transformationsschritte der Graph $G_{f^*}$ aus dem Graphen $G_f$ hervorgeht.\\
  Stelle anschließend den Funktionsterm $f^*(x)$ in Abhängigkeit von $f(x)$ auf.
\item Fertige im Koordinatensystem von Teilaufgabe a) eine möglichst genaue Skizze des Graphen der Funktion $g$ mit $g(x)=-f^*(\frac{3}{2}x)$an.
\end{enumerate}

\Aufgabe{2: (3BE+4BE+1BE)}
Gegeben ist die Funktion:
\begin{equation}
  f:x \rightarrow \begin{cases}
    x^3-4x^2+1 & \text{für $x\le 1$}\\
    \frac{3}{4}x-4 & \text{für $1<x<4$}\\
    (x-5)^2+3 & \text{für $x\ge 4$}.
  \end{cases}
\end{equation}

\begin{enumerate}[label={\alph*)}, topsep=5pt,itemsep=4ex,partopsep=1ex,parsep=1ex]
  \item Überprüfe rechnerisch, ob die Funktion $f$ an der Stelle $x=1$ stetig ist.
  \item Der zur Funktion $f$ gehörende Graph $G_f$ ist in nebenstehendem Koordinatensystem \underline{teilweise} veranschaulicht. Vervollständige den zur Funktion $f$ gehörenden Graphen in der Abbildung.
  \item Im Funktionsterm $(x-5)^2+3$ von $f$ für den Abschnitt $x\ge4$ soll der zweite Summand abgeändert werden. Modifiziere den  zweiten Summanden so, dass die Funktion $f$ an der Stelle $x=4$ stetig ist.\\

    \begin{tikzpicture}[scale=1.8]
  \begin{axis}[
    axis lines=middle,
    xmin=-3, xmax=7, ymin=-4, ymax=6,
    xlabel={$x$}, ylabel={$y$},
    xtick={-3,-2,...,7}, ytick={-5,-4,...,5},
    tick label style={font=\tiny},
    xticklabel style={anchor=north, yshift=-0.5ex},
    yticklabel style={anchor=east, xshift=-0.5ex},
    legend style={font=\tiny,legend pos=outer north east},
    axis equal,
    clip=true, % Clip the plot to axis boundaries
    every axis plot/.append style={color=blue},% Set a default color for all plots
    grid=both, % Enable grid for both axes
    grid style={line width=0.1mm, draw=gray!30}, % Customize grid line style
    major grid style={line width=0.1mm,draw=gray!30}, % Style for major grid
    minor grid style={dashed, draw=gray!30} % Style for minor grid
    ]
    % Plot the function
    \addplot+[no marks, domain=-3:1, samples=150, smooth] {x^3 - 4*x^2 + 1};
    % Plot the function
    \addplot+[no marks, domain=4:7, samples=150, smooth, color=blue] {(x-5)^2+3};
% Add an empty circle at the beginning of the domain for the first function
    % Add a circle at x=1 for the second function
    \addplot[only marks, mark=*, color=blue, mark options={fill=blue, scale=1.5}] coordinates {(1, {-2})};
    \addplot[only marks, mark=*, color=blue, mark options={fill=blue, scale=1.5}] coordinates {(4, {4})};

% Add an empty circle at the beginning of the domain for the first function
    \addplot+[only marks, mark=*, color=blue, mark options={fill=white, scale=1.5}] coordinates {(1, {(x-5)^2 + 3})};
    % Add "0" at the origin
    \node[anchor=north east, font=\tiny] at (axis cs:0,0) {0};
  \end{axis}
\end{tikzpicture}
\end{enumerate}

\newpage
\Aufgabe{3: (6BE)}
Gegeben sind die drei Funktionsgraphen (1),(2) und (3) sowie sechs Funktionsterme mit jeweils maximalem Definitionsbereich.\\

\begin{figure}[htbp]
  \centering
  % First image
  \begin{minipage}{0.3\textwidth}

\begin{tikzpicture}[scale=0.75]
  \begin{axis}[
    axis lines=middle,
    xmin=-5, xmax=6, ymin=-5, ymax=6,
    xlabel={$x$}, ylabel={$y$},
    xtick={-6,-5,...,6}, ytick={-5,-4,...,6},
    tick label style={font=\tiny},
    xticklabel style={anchor=north, yshift=-0.5ex},
    yticklabel style={anchor=east, xshift=-0.5ex},
    legend style={font=\tiny,legend pos=outer north east},
    axis equal,
    clip=true, % Clip the plot to axis boundaries
    every axis plot/.append style={color=blue},% Set a default color for all plots
    grid=both, % Enable grid for both axes
    grid style={line width=0.1mm, draw=gray!30}, % Customize grid line style
    major grid style={line width=0.1mm,draw=gray!30}, % Style for major grid
    minor grid style={dashed, draw=gray!30} % Style for minor grid
    ]
    % Plot the function
    % Plot the function for domain excluding -3 and 3
    \addplot+[no marks, domain=-7:-3-0.01, samples=150, smooth] {((2-x)*(x+2))/(x^2-9)};
    \addplot+[no marks, domain=-3+0.01:3-0.01, samples=150, smooth, color=blue] {((2-x)*(x+2))/(x^2-9)};
    \addplot+[no marks, domain=3+0.01:7, samples=150, smooth, color=blue] {((2-x)*(x+2))/(x^2-9)};
  % Vertical asymptotes at x = 3 and x = -3
    \addplot+[no marks, domain=-3-0.01:-3+0.01, samples=15, smooth, color=blue, style=dashed] {((2-x)*(x+2))/(x^2-9)};
    \addplot+[no marks, domain=3-0.01:3+0.01, samples=15, smooth, color=blue, style=dashed] {((2-x)*(x+2))/(x^2-9)};
    \addplot+[no marks, domain=3-0.01:3+0.01, samples=15, smooth, color=blue, style=dashed] {((2-x)*(x+2))/(x^2-9)};
% Horizontal asymptote at y = -1
    \addplot[dashed, color=blue] coordinates {(-5, -1) (5, -1)};

    % Add "(1)" in the upper left corner
    \node[anchor=north west, font=\large] at (axis cs:-6,6) {(1)};
  \end{axis}
\end{tikzpicture}
\end{minipage}%
  \hfill
  % Second image
  \begin{minipage}{0.3\textwidth}
\begin{tikzpicture}[scale=0.75]
  \begin{axis}[
    axis lines=middle,
    xmin=-5, xmax=6, ymin=-5, ymax=6,
    xlabel={$x$}, ylabel={$y$},
    xtick={-6,-5,...,6}, ytick={-5,-4,...,6},
    tick label style={font=\tiny},
    xticklabel style={anchor=north, yshift=-0.5ex},
    yticklabel style={anchor=east, xshift=-0.5ex},
    legend style={font=\tiny,legend pos=outer north east},
    axis equal,
    clip=true, % Clip the plot to axis boundaries
    every axis plot/.append style={color=blue},% Set a default color for all plots
    grid=both, % Enable grid for both axes
    grid style={line width=0.1mm, draw=gray!30}, % Customize grid line style
    major grid style={line width=0.1mm,draw=gray!30}, % Style for major grid
    minor grid style={dashed, draw=gray!30} % Style for minor grid
    ]
    % Plot the function
    % Plot the function for domain excluding -3 and 3
    \addplot+[no marks, domain=-7:1-0.01, samples=150, smooth] {(1-0.5*x)+(1/(x-1))};
    \addplot+[no marks, domain=1+0.01:7, samples=150, smooth, color=blue] {(1-0.5*x)+(1/(x-1))};
    \node[anchor=north west, font=\large] at (axis cs:-6,6) {(2)};

    \addplot+[no marks, domain=1-0.01:1+0.01, samples=15, smooth, color=blue, style=dashed] {(1-0.5*x)+(1/(x-1))};

    \addplot+[no marks, domain=-7:7, samples=150, smooth, color=blue, style=dashed] {(1-0.5*x)};
  \end{axis}
\end{tikzpicture}
\end{minipage}%
  \hfill
  % Third image
\begin{minipage}{0.3\textwidth}

\begin{tikzpicture}[scale=0.75]
  \begin{axis}[
    axis lines=middle,
    xmin=-5, xmax=9, ymin=-5, ymax=8,
    xlabel={$x$}, ylabel={$y$},
    xtick={-6,-5,...,9}, ytick={-5,-4,...,8},
    tick label style={font=\tiny},
    xticklabel style={anchor=north, yshift=-0.5ex},
    yticklabel style={anchor=east, xshift=-0.5ex},
    legend style={font=\tiny,legend pos=outer north east},
    axis equal,
    clip=true, % Clip the plot to axis boundaries
    every axis plot/.append style={color=blue},% Set a default color for all plots
    grid=both, % Enable grid for both axes
    grid style={line width=0.1mm, draw=gray!30}, % Customize grid line style
    major grid style={line width=0.1mm,draw=gray!30}, % Style for major grid
    minor grid style={dashed, draw=gray!30} % Style for minor grid
    ]
    % Plot the function for domain excluding -3 and 3
    \addplot+[no marks, domain=-7:2-0.01, samples=150, smooth] {(3+x)/(x-2)};
    \addplot+[no marks, domain=2+0.01:9, samples=150, smooth, color=blue] {(3+x)/(x-2)};
    \node[anchor=north west, font=\large] at (axis cs:-5,8) {(3)};

    \addplot+[no marks, domain=2-0.01:2+0.01, samples=15, smooth, color=blue, style=dashed] {(3+x)/(x-2)};

% Horizontal asymptote at y = -1
    \addplot[dashed, color=blue] coordinates {(-7, 1) (9, 1)};
  \end{axis}
\end{tikzpicture}
\end{minipage}%
\end{figure}


    \begin{tabular}{ p{4cm} | p{4cm} | p{4cm} | }
      %\cline{2-3}
      \hline
      \multicolumn{1}{|l|}{
        \rule[-0.7em]{0pt}{2em}
        $f(x)=\frac{x-2}{3+x}$}&
                $g(x)=\frac{x^2-4}{(x+3)(x-3)}$&
                $h(x)=\frac{(2-x)(x+2)}{x^2-9}\\ \hline
      \multicolumn{1}{|l|}{
        \rule[-0.7em]{0pt}{2em}
        $i(x)=1-0,5x+\frac{1}{x-1}$}&
                $k(x)=\frac{3+x}{x-2}$&
                $m(x)=1+0,5x+\frac{1}{x-1}$\\ \hline
    \end{tabular}\\

Ordne jedem Graphen den passenden Funktionsterm zu.\\
Bergründe und erläutere deine Entscheidung, ohne dabei Funktionswerte zu benutzen.

\Aufgabe{4: (4BE+5BE+4BE+2BE)}
Gegeben ist die Funktion $f$ mit $f(x)=\frac{x^2+4x+4}{2x-2}$ und maximalem Definitionsbereich $D_f$.
\begin{enumerate}[label={\alph*)}, topsep=5pt,itemsep=4ex,partopsep=1ex,parsep=1ex]
  \item Gib die \underline{maximale Definitionsmenge $D_f$} an. Bestimme das Verhalten des Graphen $G_f$ in der Nähe der Definitionslücke und benenne auch deren Art. Gib abschließend die Gleichung der zugehörigen Asymptote $a_1$ an.
  \item Ermittle die Koordinaten der Schnittpunkte des Graphen der Funktion $f$ mit den Koordinatenachsen. Gib zudem die Vielfachheit der Nullstelle(n) an.
  \item Zeige, dass $h(x)=0,5x+2,5+\frac{9}{2x-2}$ eine äquivalente Darstellung von $f$ ist und gib ausgehend von dieser Darstellung die Art und die Gleichung einer weiteren, zum Graphen der Funktion $f$ gehörigen Asympote $a_2$ an.
  \item Begründe mithilfe von Funktion $h$ aus Teilaufgabe 4.c), dass sich der Graph $G_f$ für $x\rightarrow +\infty$ von oben an die Asymptote $a_2$ annähert.
\end{enumerate}

\vspace{5mm}
\begin{center}
Viel Erfolg!
\end{center}

\end{document}
