%compile with pdflatex on papeeria

\documentclass[a4paper,12pt]{article}


\usepackage{fontawesome}
\usepackage{fancyhdr}
\usepackage{fancyheadings}
\usepackage[ngerman,german]{babel}
\usepackage{german}
\usepackage[utf8]{inputenc}
%\usepackage[latin1]{inputenc}
\usepackage[active]{srcltx}
%\usepackage{svg}
%\usepackage{algorithm}
%\usepackage[noend]{algorithmic}
\usepackage{eurosym}
\usepackage{amsmath}
\usepackage{amssymb}
\usepackage{amsthm}
\usepackage{bbm}
\usepackage{enumerate}
\usepackage{graphicx}
\usepackage{ifthen}
\usepackage{listings}
\usepackage{enumitem}
%\usepackage{struktex}
\usepackage{hyperref}
\usepackage{tikz}
\usepackage{float}
\usepackage{subcaption}
\usepackage{array}
\captionsetup{compatibility=false}
\captionsetup[subfigure]{labelformat=empty}

\usepackage{pgfplots}
\pgfplotsset{compat=1.15}
\usepackage{mathrsfs}
\usetikzlibrary{arrows}

\definecolor{ccqqqq}{rgb}{0.8,0,0}
\definecolor{kolorwykresu}{rgb}{0.07,0.04,0.56}

\pagenumbering{gobble}

\usepackage{tabularray}
\usepackage{multirow}
\usepackage{booktabs,tabularx}

%\DeclareMathSymbol{\shortminus}{\mathbin}{AMSa}{"39}

\renewcommand\tabularxcolumn[1]{m{#1}}% for vertical centering text in X column

\newcolumntype{L}[1]{>{\raggedright\let\newline\\\arraybackslash\hspace{0pt}}m{#1}}
\newcolumntype{C}[1]{>{\centering\let\newline\\\arraybackslash\hspace{0pt}}m{#1}}
\newcolumntype{R}[1]{>{\raggedleft\let\newline\\\arraybackslash\hspace{0pt}}m{#1}}

\newcolumntype{Y}{>{\centering\arraybackslash}X}

%%%%%%%%%%%%%%%%%%%%%%%%%%%%%%%%%%%%%%%%%%%%%%%%%%%%%%
%%%%%%%%%%%%%% EDIT THIS PART %%%%%%%%%%%%%%%%%%%%%%%%
%%%%%%%%%%%%%%%%%%%%%%%%%%%%%%%%%%%%%%%%%%%%%%%%%%%%%%
\newcommand{\Fach}{1. Schulaufgabe aus der Mathematik}
\newcommand{\Name}{}
\newcommand{\datum}{}
\newcommand{\Matrikelnummer}{}
\newcommand{\Semester}{}
\newcommand{\Uebungsblatt}{} %  <-- UPDATE ME
%%%%%%%%%%%%%%%%%%%%%%%%%%%%%%%%%%%%%%%%%%%%%%%%%%%%%%
%%%%%%%%%%%%%%%%%%%%%%%%%%%%%%%%%%%%%%%%%%%%%%%%%%%%%%

\setlength{\parindent}{0em}
\topmargin -1.0cm
\oddsidemargin 0cm
\evensidemargin 0cm
\setlength{\textheight}{9.2in}
\setlength{\textwidth}{6.0in}

%%%%%%%%%%%%%%%
%% Aufgaben-COMMAND
\newcommand{\Aufgabe}[1]{
  {
  \vspace*{0.5cm}
  \textsf{\textbf{Aufgabe #1}}
  \vspace*{0.2cm}
  
  }
}
%%%%%%%%%%%%%%
\hypersetup{
    pdftitle={\Fach{}: Übungsblatt \Uebungsblatt{}},
    pdfauthor={\Name},
    pdfborder={0 0 0}
}

\lstset{ %
language=java,
basicstyle=\footnotesize\tt,
showtabs=false,
tabsize=2,
captionpos=b,
breaklines=true,
extendedchars=true,
showstringspaces=false,
flexiblecolumns=true,
}

\newcommand*{\quadratbox}{\textbf{\fbox{\phantom{\huge{?}}}}}%

\title{Übungsblatt \Uebungsblatt{}}
\author{\Name{}}

\begin{document}

\fancyhead{}
\fancyhead[C]{\includegraphics[height=1.5cm]{lukasLogoV2.png}
\vspace{1cm}
}

\thispagestyle{fancy}

\lhead{
%\vspace{1cm}
  \sf \LARGE \Fach{} %\small \Name{} - \Matrikelnummer{}
}
\rhead{\sf \Semester{}   \datum{}}


\vspace*{0.2cm}

\vspace{2cm}
Alle Lösungen müssen mit Nebenrechnungen und Begründungen nachvollziehbar sein!

%\rhead{\sf \Semester{} }
%\vspace*{0.2cm}

%\begin{center}
%%\LARGE \sf \textbf{Übungsblatt \Uebungsblatt{}}
%\end{center}
%\vspace*{0.2cm}

%%%%%%%%%%%%%%%%%%%%%%%%%%%%%%%%%%%%%%%%%%%%%%%%%%%%%%
%% Insert your solutions here %%%%%%%%%%%%%%%%%%%%%%%%
%%%%%%%%%%%%%%%%%%%%%%%%%%%%%%%%%%%%%%%%%%%%%%%%%%%%%%

\vspace{1cm}
  Name: \underline{\hspace{7cm}}
  \hfill
  Datum: \underline{\hspace{4cm}}

%\vspace{0.8cm}

%\textbf{Hinweise:} Der Lösungsweg muss nachvollziehbar sein. Arbeitszeit \textbf{45 Minuten}.
%\vspace{0,5cm}Die Rechenwege müssen nachvollziehbar sein!
%
%\vspace{0,5cm} {TEIL A} - ohne Hilfsmittel - Bearbeitungszeit 30 Minuten
%\vspace {0.8cm}
% 
%GEOMETRIE

%\begin{center}
%  \begin{tblr}{
%      width=1\linewidth,
%      colspec = {Q[c,5em]Q[c,2em]Q[c,2em]Q[c,2em]Q[c,2em]Q[c,2em]Q[c,2em]Q[c,2em]Q[c,2em]Q[c,2em]Q[c,2em]Q[c,5em]Q[c,3em]},
%      rowspec = {Q[m]Q[m]Q[m]Q[m]Q[m]Q[m]Q[m]Q[m]Q[m]Q[m]Q[m]Q[m]Q[m]},
%      colsep = 0mm,
%      %row{1} = {2em,azure2,fg=white,font=\large\bfseries\sffamily},
%      row{1} = {2em,font=\large\bfseries\sffamily},
%      hlines, vlines,
%    }
%    \textbf{Aufgabe} & \textbf{1} & \textbf{2} & \textbf{3} & 
%    \textbf{4} & \textbf{5} & \textbf{6}& \textbf{7}& \textbf{8}& \textbf{9}& \textbf{10}&
%    \textbf{Gesamt} & \textbf{Note}\\
%    {Mögliche \\ Punkte} & 4 & 4 & 4 & 8 & 13 & 13 & 3 & 6 & 7 & 3 & 65 \\
%    {Erreichte \\ Punkte} &  &  &  &  & & & & & & & & \\
%  \end{tblr}
%\end{center}

%\vspace{0.3cm}
%\newpage\null\thispagestyle{empty}\newpage
%\newpage
%\vspace{1,5cm} {TEIL A} - ohne Hilfsmittel. Bearbeitungszeit 25 min.
%\vspace {0,2cm}

\Aufgabe{1: (2BE+3BE+3BE+2BE)}
Eine 1-Liter-Flasche frische Vollmilch enthält ungefähr 10mg Vitamin C, welches sich aber unter Einfluss von Licht zersetzt. Pro Stunde nimmt der Vitamin-C-Gehalt der Milch dabei um 6\% ab.\\

\begin{enumerate}[label={\alph*)}, topsep=5pt,itemsep=4ex,partopsep=1ex,parsep=1ex]
  \item Beschreibe die Abhahme des Vitamin-C-Gehalts durch einen Funktionsterm $G(n)$, der die verbleibende Vitamin-C-Menge nach $n$ verstricenen Stunden wiedergibt.  
  \item Berechne, nach wie vielen Stunden nur noch die Hälfte der anfänglichen Menge an Vitamin C in der Milch vorhanden ist.
  \item Bestimme den Prozentsatz, um den der Vitamin-C-Gehalt in der Milch pro Stunde abnehmen müsste, damit die verbleibende Menge an Vitamin C nach 5,5 Stunden noch 6mg beträgt.
  \item Beurteilen Sie folgen Aussage: ,,Es gibt keinen Zeitpunkt an dem sich das Vitamin C vollständig aufgelöst hat.''
\end{enumerate}

\Aufgabe{2: (4BE+3BE)}
Löse folgende Exponentialgleichung und gib alle Zwischenschritte an. Runde dein Ergebnis auf zwei Dezimalstellen.

\begin{enumerate}[label={\alph*)}, topsep=5pt,itemsep=4ex,partopsep=1ex,parsep=1ex]
  \item $2^{x+3} + 2^{x-1} = 4$

  \item $4^{(2x)^2} = 8$

\end{enumerate}

\newpage
\Aufgabe{3: (2BE+2BE)}

\begin{enumerate}[label={\alph*)}, topsep=5pt,itemsep=4ex,partopsep=1ex,parsep=1ex]
  \item Bestimme die Variable $a$: $ log_2 a = 4 $
  \item Schreibe den folgenden Ausdruck als Exponentialgleichung und berechne $x$: $ log_{a^{\frac{1}{2}}} a^{1,5} = x $

\end{enumerate}

\Aufgabe{4: (5BE+2BE)}

\begin{enumerate}[label={\alph*)}, topsep=5pt,itemsep=4ex,partopsep=1ex,parsep=1ex]
  \item Bestimme zu den beiden untenstehenden Graphen $f(x)$ und $g(x)$ jeweils eine zugehörige Funktionsvorschrift.\\
\raisebox{-10mm}{\includegraphics[keepaspectratio = true, scale = 1.2] {20241127.png}}

  \item Zeichne die Funktion $h(x) = 2 \cdot {1,5}^x$ in das obenstehende Koordinatensystem ein.

\end{enumerate}
\end{document}
